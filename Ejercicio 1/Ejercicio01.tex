\documentclass[fleqn]{article}

%\pgfplotsset{compat=1.17}

\usepackage{mathexam}
\usepackage{amsmath}
\usepackage{amsfonts}
\usepackage{graphicx}
\usepackage{systeme}
\usepackage{microtype}
\usepackage{multirow}
\usepackage{pgfplots}
\usepackage{listings}
\usepackage{tikz}
\usepackage{dsfont} %Numeros reales, naturales...
\usepackage{cancel}
\usepackage{babel}

%\graphicspath{{images/}}
\newcommand*{\QED}{\hfill\ensuremath{\square}}

%Estructura de ecuaciones
\setlength{\textwidth}{15cm} \setlength{\oddsidemargin}{5mm}
\setlength{\textheight}{23cm} \setlength{\topmargin}{-1cm}



\author{David García Curbelo}
\title{Teoría de Números y Criptografía}

\pagestyle{empty}


\def\R{\mathds{R}}
\def\Z{\mathds{Z}}
\def\N{\mathds{N}}

\def\sup{$^2$}

\def\next{\quad \Rightarrow \quad}


\begin{document}
    \begin{center}
        \LARGE{\textbf{Ejercicio 1}} \\
        \Large{David García Curbelo} \\
    \end{center}

    \vspace{1cm}
    
    Dado tu número $n$ = 45352581: \\ 

    \textbf{Apartado I.\textit{Mientras $n$ sea múltiplo de 2, 3, 5, 7 u 11 le sumas 1. De forma que tu nuevo $n$ no tenga esos 
    divisores primos.}}
    \begin{enumerate}
        \item[$\bullet$] Vemos que el número $45352581$ es impar, luego no es múltiplo de 2. 
        \item[$\bullet$] Pero, como la suma de sus dígitos es $33$ (múltiplo de 3), actualizamos el valor de $n$ añadiéndole una unidad,
                obteniendo $45352582$ . 
        \item[$\bullet$] Repetimos el proceso. Ahora $45352582$ es múltiplo de 2, luego sumamos 1 y actualizamos su valor a $45352583$ . 
        \item[$\bullet$] Vemos que la suma de los dígitos de $45352583$ es $34$ el cual no es múltiplo de 3. 
        \item[$\bullet$] Como $45352583$ no acaba ni en 5 ni en 0, dicho valor no es múltiplo de 5. 
        \item[$\bullet$] Como $45352583$ módulo 7 es congruente con 3, $45352583$ no es múltiplo de 7. 
        \item[$\bullet$] Como $45352583$ módulo 11 es congruente con 1, $45352583$ no es múltiplo de 11.
    \end{enumerate}

    Hemos conseguido así el número buscado, $n = 45352583$.\\ \\
     
    \newpage
    \textbf{Apartado II.\textit{Calcula $a^{n-1} \pmod{n}$, para cada uno de esas cinco bases usando sucesivamente 
    el algoritmo izda-drcha y de drcha-izda.}}

    Para ambos procesos necesitaremos saber la representación de $n-1$ en base 2, para el que tenemos que 
    $45352582_{10} = 10101101000000011010000110_{2}$. Como dicho número tiene 26 cifras, cada proceso constará
    de 26 iteraciones, siendo la última el resultado de la operación $a^{n-1} \pmod{n}$ para los distintos
    valores de $a$.\\ 

    \textbf{izda-drcha:}

    %\begin{center}
        \begin{tabular}{c}
            Base 2 \\
        \end{tabular}
        \begin{tabular}{c | c}
            Iteración & Acumulado \\ \hline
            0 & 1 \\
            1 & 2   \\
            2 & 4   \\
            3 & 32      \\
            4 & 1024        \\
            5 & 2097152     \\
            6 & 4901941     \\
            7 & 2574340     \\
            8 & 24431701    \\
            9 & 37951484    \\
            10 & 6243980     \\
            11 & 28969616    \\
            12 & 32744959    \\
            13 & 41853729    \\
            14 & 1936709     \\
            15 & 1726249     \\
            16 & 42935389    \\
            17 & 16426326    \\
            18 & 36409185    \\
            19 & 16353216    \\
            20 & 45284451    \\
            21 & 16005958    \\
            22 & 27376886    \\
            23 & 2964308     \\
            24 & 27220062    \\
            25 & 10671452    \\
            26 & 5401134     \\ \hline
            & $ 2^{n-1} \pmod{n} \equiv  5401134 $
        \end{tabular}
        \begin{tabular}{c}
            Base 3 \\
        \end{tabular}
        \begin{tabular}{c | c}
            Iteración & Acumulado \\ \hline
            0 & 1 \\
            1 & 3   \\
            2 & 9   \\
            3 & 243     \\
            4 & 59049   \\
            5 & 29259113    \\
            6 & 11292172    \\
            7 & 7525116     \\
            8 & 44218304    \\
            9 & 26775297    \\
           10 & 40650583    \\
           11 & 9371079     \\
           12 & 8109681     \\
           13 & 11498886    \\
           14 & 11966488    \\
           15 & 45258948    \\
           16 & 43394118    \\
           17 & 34167664    \\
           18 & 33986041    \\
           19 & 1181461     \\
           20 & 33647530    \\
           21 & 17298295    \\
           22 & 18877819    \\
           23 & 24620948    \\
           24 & 16028215    \\
           25 & 26671429    \\
           26 & 19036530    \\ \hline
           & $ 3^{n-1} \pmod{n} \equiv 19036530 $
        \end{tabular}

        \begin{tabular}{c}
            Base 5 \\
        \end{tabular}
        \begin{tabular}{c | c}
            Iteración & Acumulado \\ \hline
            0 & 1 \\
            1 & 5   \\
            2 & 25  \\
            3 & 3125    \\
            4 & 9765625     \\
            5 & 9835959     \\
            6 & 23733320    \\
            7 & 14353170    \\
            8 & 33185227    \\
            9 & 24638670    \\
           10 & 7540295     \\
           11 & 9771990     \\
           12 & 20245114    \\
           13 & 14642594    \\
           14 & 43749178    \\
           15 & 5721504     \\
           16 & 23832084    \\
           17 & 36544685    \\
           18 & 27160596    \\
           19 & 37821484    \\
           20 & 10726414    \\
           21 & 37082453    \\
           22 & 44314341    \\
           23 & 6257820     \\
           24 & 32829606    \\
           25 & 11061821    \\
           26 & 29804810    \\ \hline
           & $ 5^{n-1} \pmod{n} \equiv 29804810 $
        \end{tabular}
        \begin{tabular}{c}
            Base 7 \\
        \end{tabular}
        \begin{tabular}{c | c}
            Iteración & Acumulado \\ \hline
            0 & 1 \\
            1 & 7   \\
            2 & 49  \\
            3 & 16807   \\
            4 & 10359751    \\
            5 & 15063881    \\
            6 & 14246235    \\
            7 & 44415909    \\
            8 & 41896404    \\
            9 & 28559169    \\
           10 & 20398933    \\
           11 & 33094120    \\
           12 & 27187659    \\
           13 & 22911122    \\
           14 & 7960948     \\
           15 & 41170261    \\
           16 & 44361304    \\
           17 & 38890192    \\
           18 & 24907161    \\
           19 & 34522948    \\
           20 & 30941719    \\
           21 & 31520605    \\
           22 & 25095178    \\
           23 & 41152615    \\
           24 & 42116794    \\
           25 & 44647582    \\
           26 & 7452904     \\ \hline
           & $ 7^{n-1} \pmod{n} \equiv 7452904 $
        \end{tabular}

        \begin{tabular}{c}
            Base 11 \\
        \end{tabular}
        \begin{tabular}{c | c}
            Iteración & Acumulado \\ \hline
            0 & 1 \\
            1 & 11  \\
            2 & 121     \\
            3 & 161051  \\
            4 & 41099708    \\
            5 & 18758252    \\
            6 & 39784534    \\
            7 & 7869852     \\
            8 & 32351398    \\
            9 & 11165071    \\
            10 & 24998844   \\
            11 & 25231382   \\
            12 & 39831069   \\
            13 & 22097604   \\
            14 & 7087930    \\
            15 & 17450229   \\
            16 & 30945672   \\
            17 & 37816540   \\
            18 & 33735176   \\
            19 & 42375724   \\
            20 & 21550596   \\
            21 & 5175263    \\
            22 & 16407855   \\
            23 & 10991810   \\
            24 & 44127982   \\
            25 & 28686620   \\
            26 & 2015230    \\ \hline
            & $ 11^{n-1} \pmod{n} \equiv  2015230 $
        \end{tabular}
    %\end{center}

    \newpage
    \textbf{drcha-izda:} %%%%%%%%%%%%%%%%%%%%%%%%%%%%%%%%%%%%%%%%%%%%%%%%%%%%%%%%%%%%%%%

    \begin{center}
        \begin{tabular}{c}
            Base 2 \\
        \end{tabular}
        \begin{tabular}{c | c | c}
            Iteración & Exponente & Acumulado \\ \hline
            0 & 45352582 & 1 \\
            1 & 22676291 & 1 \\
            2 & 11338145 & 4 \\
            3 & 5669072 & 64 \\
            4 & 2834536 & 64 \\
            5 & 1417268 & 64 \\
            6 & 708634 & 64 \\
            7 & 354317 & 64 \\
            8 & 177158 & 3644558 \\
            9 & 88579 & 3644558 \\
            10 & 44289 & 4282517 \\
            11 & 22144 & 31320796 \\
            12 & 11072 & 31320796 \\
            13 & 5536 & 31320796 \\
            14 & 2768 & 31320796 \\
            15 & 1384 & 31320796 \\
            16 & 692 & 31320796 \\
            17 & 346 & 31320796 \\
            18 & 173 & 31320796 \\
            19 & 86 & 32659444 \\
            20 & 43 & 32659444 \\
            21 & 21 & 15341136 \\
            22 & 10 & 13749273 \\
            23 & 5 & 13749273 \\
            24 & 2 & 25094767 \\
            25 & 1 & 25094767 \\
            26 & 0 & 5401134 \\ \hline
            && $ 2^{n-1} \pmod{n} \equiv  5401134 $
        \end{tabular}
        \newpage
        \begin{tabular}{c}
            Base 3 \\
        \end{tabular}
        \begin{tabular}{c | c | c}
            Iteración & Exponente & Acumulado \\ \hline
            0 & 45352582 & 1 \\
            1 & 22676291 & 1 \\
            2 & 11338145 & 9 \\
            3 & 5669072 & 729 \\
            4 & 2834536 & 729 \\
            5 & 1417268 & 729 \\
            6 & 708634 & 729 \\
            7 & 354317 & 729 \\
            8 & 177158 & 24311052 \\
            9 & 88579 & 24311052 \\
            10 & 44289 & 35459567 \\
            11 & 22144 & 35739659 \\
            12 & 11072 & 35739659 \\
            13 & 5536 & 35739659 \\
            14 & 2768 & 35739659 \\
            15 & 1384 & 35739659 \\
            16 & 692 & 35739659 \\
            17 & 346 & 35739659 \\
            18 & 173 & 35739659 \\
            19 & 86 & 41990270 \\
            20 & 43 & 41990270 \\
            21 & 21 & 21307220 \\
            22 & 10 & 37476107 \\
            23 & 5 & 37476107 \\
            24 & 2 & 45273497 \\
            25 & 1 & 45273497 \\
            26 & 0 & 19036530 \\
            && $ 3^{n-1} \pmod{n} \equiv 19036530 $
        \end{tabular}

        \newpage
        \begin{tabular}{c}
            Base 5 \\
        \end{tabular}
        \begin{tabular}{c | c | c}
            Iteración & Exponente & Acumulado \\ \hline
            0 & 45352582 & 1 \\
            1 & 22676291 & 1 \\
            2 & 11338145 & 25 \\
            3 & 5669072 & 15625 \\
            4 & 2834536 & 15625 \\
            5 & 1417268 & 15625 \\
            6 & 708634 & 15625 \\
            7 & 354317 & 15625 \\
            8 & 177158 & 41450728 \\
            9 & 88579 & 41450728 \\
            10 & 44289 & 5511674 \\
            11 & 22144 & 3328245 \\
            12 & 11072 & 3328245 \\
            13 & 5536 & 3328245 \\
            14 & 2768 & 3328245 \\
            15 & 1384 & 3328245 \\
            16 & 692 & 3328245 \\
            17 & 346 & 3328245 \\
            18 & 173 & 3328245 \\
            19 & 86 & 24581849 \\
            20 & 43 & 24581849 \\
            21 & 21 & 25673283 \\
            22 & 10 & 21813943 \\
            23 & 5 & 21813943 \\
            24 & 2 & 43254326 \\
            25 & 1 & 43254326 \\
            26 & 0 & 29804810 \\ \hline
            && $ 5^{n-1} \pmod{n} \equiv 29804810 $
        \end{tabular}
        \newpage
        \begin{tabular}{c}
            Base 7 \\
        \end{tabular}
        \begin{tabular}{c | c | c}
            Iteración & Exponente & Acumulado \\ \hline
            0 & 45352582 & 1 \\
            1 & 22676291 & 1 \\
            2 & 11338145 & 49 \\
            3 & 5669072 & 117649 \\
            4 & 2834536 & 117649 \\
            5 & 1417268 & 117649 \\
            6 & 708634 & 117649 \\
            7 & 354317 & 117649 \\
            8 & 177158 & 28491703 \\
            9 & 88579 & 28491703 \\
            10 & 44289 & 4829743 \\
            11 & 22144 & 17486073 \\
            12 & 11072 & 17486073 \\
            13 & 5536 & 17486073 \\
            14 & 2768 & 17486073 \\
            15 & 1384 & 17486073 \\
            16 & 692 & 17486073 \\
            17 & 346 & 17486073 \\
            18 & 173 & 17486073 \\
            19 & 86 & 24761580 \\
            20 & 43 & 24761580 \\
            21 & 21 & 18994483 \\
            22 & 10 & 22336683 \\
            23 & 5 & 22336683 \\
            24 & 2 & 32942235 \\
            25 & 1 & 32942235 \\
            26 & 0 & 27704450 \\ \hline
            && $ 7^{n-1} \pmod{n} \equiv 27704450 $
        \end{tabular}

        \newpage
        \begin{tabular}{c}
            Base 11 \\
        \end{tabular}
        \begin{tabular}{c | c | c}
            Iteración & Exponente & Acumulado \\ \hline
            0 & 45352582 & 1 \\
            1 & 22676291 & 1 \\
            2 & 11338145 & 121 \\
            3 & 5669072 & 1771561 \\
            4 & 2834536 & 1771561 \\
            5 & 1417268 & 1771561 \\
            6 & 708634 & 1771561 \\
            7 & 354317 & 1771561 \\
            8 & 177158 & 23163409 \\
            9 & 88579 & 23163409 \\
            10 & 44289 & 18945097 \\
            11 & 22144 & 43014292 \\
            12 & 11072 & 43014292 \\
            13 & 5536 & 43014292 \\
            14 & 2768 & 43014292 \\
            15 & 1384 & 43014292 \\
            16 & 692 & 43014292 \\
            17 & 346 & 43014292 \\
            18 & 173 & 43014292 \\
            19 & 86 & 11213684 \\
            20 & 43 & 11213684 \\
            21 & 21 & 37482594 \\
            22 & 10 & 37985641 \\
            23 & 5 & 37985641 \\
            24 & 2 & 18608191 \\
            25 & 1 & 18608191 \\
            26 & 0 & 26386779 \\ \hline
            && $ 11^{n-1} \pmod{n} \equiv 26386779 $
        \end{tabular}
    \end{center} 


    \newpage
    \textbf{Apartado III.\textit{¿Es $n$ un posible primo de Fermat para alguna de ellas? ¿Es $n$ un pseudoprimo para alguna de 
    ellas?}} 


    \begin{enumerate}
        \item[$\bullet$] Para que $n$ sea un posible primo de Fermat para alguna base, n debe pasar el test para dicha base. Como 
                        hemos visto en el apartado anterior, ninguna de las cinco bases mencionadas cumplen $ 1 \equiv a^{n-1} \pmod{n} $,
                        Luego $n$ no es primo de Fermat para ninguna de dichas bases.
        \item[$\bullet$] Para que $n$ sea un pseudoprimo para alguna base, n debe de pasar el test para dicha base y además ser compuesto.
                        Como no se cumple para ninguna base, no tenemos la condición de que $n$ sea un pseudoprimo para ninguna de las bases. 
    \end{enumerate}

\end{document}