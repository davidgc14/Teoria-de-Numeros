\documentclass[fleqn]{article}

%\pgfplotsset{compat=1.17}

\usepackage{mathexam}
\usepackage{amsmath}
\usepackage{amsfonts}
\usepackage{graphicx}
\usepackage{systeme}
\usepackage{microtype}
\usepackage{multirow}
\usepackage{pgfplots}
\usepackage{listings}
\usepackage{tikz}
\usepackage{dsfont} %Numeros reales, naturales...
\usepackage{cancel}
\usepackage{babel}

%\graphicspath{{images/}}
\newcommand*{\QED}{\hfill\ensuremath{\square}}

%Estructura de ecuaciones
\setlength{\textwidth}{15cm} \setlength{\oddsidemargin}{5mm}
\setlength{\textheight}{23cm} \setlength{\topmargin}{-1cm}



\author{David García Curbelo}
\title{Teoría de Números y Criptografía}

\pagestyle{empty}


\def\R{\mathds{R}}
\def\Z{\mathds{Z}}
\def\N{\mathds{N}}

\def\sup{$^2$}

\def\next{\quad \Rightarrow \quad}


\begin{document}
    \begin{center}
        \LARGE{\textbf{Ejercicio 5}} \\
        \Large{David García Curbelo} \\
    \end{center}

    \vspace{1cm}
    
    Dado tu número $n = 11781277$ de la lista publicada para este ejercicio: \\ 


    \textbf{Apartado I. \textit{Factoriza $n$ aplicando el método $\rho$ de Polard. ¿Cuántas iteraciones necesitas?}} 
    
    Aplicando el método $\rho$ de Polard, tenemos que para $n = 11781277$ obtenemos el primer factor primo 2591
    en 58 iteraciones, como se puede ver en la tabla de la página siguiente. Por ello tenemos que nuestro número $n$ se
    nos queda factorizado como producto de dos primos $n = 2591 \cdot 4547$ (el segundo sabemos que es primo por estar 
    presente en la tabla de primos menores de 5 cifras).
    \begin{center}
        \begin{tabular}{| c | c | c | c |}
            \hline Paso & $x$ & $y$ & mcd \\ \hline
            %0 & 1 & 1 & - \\ \hline
            1 & 2 & 5 & 1 \\
            2 & 5 & 677 & 1 \\
            3 & 26 & 6219991 & 1 \\
            4 & 677 & 5601822 & 1 \\
            5 & 458330 & 2597501 & 1 \\
            6 & 6219991 & 807607 & 1 \\
            7 & 11687876 & 8322365 & 1 \\
            8 & 5601822 & 1643871 & 1 \\
            9 & 584194 & 5993347 & 1 \\
            10 & 2597501 & 7806461 & 1 \\
            11 & 3701149 & 5941709 & 1 \\
            12 & 807607 & 341798 & 1 \\
            13 & 5790453 & 3285327 & 1 \\
            14 & 8322365 & 8668694 & 1 \\
            15 & 8711090 & 7679181 & 1 \\
            16 & 1643871 & 165755 & 1 \\
            17 & 5015321 & 7757667 & 1 \\
            18 & 5993347 & 9313459 & 1 \\
            19 & 1847739 & 356366 & 1 \\
            20 & 7806461 & 11451286 & 1 \\
            21 & 10307054 & 11732626 & 1 \\
            22 & 5941709 & 8156866 & 1 \\
            23 & 1588435 & 7512258 & 1 \\
            24 & 341798 & 11098752 & 1 \\
            25 & 2730073 & 4268072 & 1 \\
            26 & 3285327 & 7145982 & 1 \\
            27 & 3698488 & 10781441 & 1 \\
            28 & 8668694 & 6061976 & 1 \\
            29 & 4728818 & 3273655 & 1 \\
            30 & 7679181 & 10069807 & 1 \\
            31 & 5435394 & 11002644 & 1 \\
            32 & 165755 & 686311 & 1 \\
            33 & 782062 & 8650796 & 1 \\
            34 & 7757667 & 10952160 & 1 \\
            35 & 7142119 & 3070436 & 1 \\
            36 & 9313459 & 7757789 & 1 \\
            37 & 6598961 & 907115 & 1 \\
            38 & 356366 & 5241407 & 1 \\
            39 & 6341174 & 10800383 & 1 \\
            40 & 11451286 & 9833373 & 1 \\
            41 & 11498048 & 9739199 & 1 \\
            42 & 11732626 & 307635 & 1 \\
            43 & 10664402 & 11520196 & 1 \\
            44 & 8156866 & 4229214 & 1 \\
            45 & 7397659 & 3970356 & 1 \\
            46 & 7512258 & 540908 & 1 \\
            47 & 6924677 & 1648763 & 1 \\
            48 & 11098752 & 1744952 & 1 \\
            49 & 8683046 & 11158052 & 1 \\
            50 & 4268072 & 8127891 & 1 \\
            51 & 4255522 & 1144281 & 1 \\
            52 & 7145982 & 4961988 & 1 \\
            53 & 8964877 & 5443366 & 1 \\
            54 & 10781441 & 5449568 & 1 \\
            55 & 7110893 & 11084302 & 1 \\
            56 & 6061976 & 6817211 & 1 \\
            57 & 6432581 & 10456312 & 1 \\
            58 & 3273655 & 6610863 & 2591 \\ \hline
        \end{tabular}
    \end{center}

    \newpage
    \textbf{Apartado II. \textit{Sea $p_1$ el mayor de sus factores primos y $p_2$ el siguiente primo.\\ 
    Calcula las partes enteras de $\sqrt{p_1}$ y $\sqrt{p_2}$ con el algoritmo entero.}}

    Tenemos que $p_1 = 4547$ y $p_2 = 2591$. Por lo tanto, como ambos son impares, para proceder con el algoritmo
    consideramos los primeros $a$ como $a_{p_1} = (4547 + 1) / 2 = 2274$ y $a_{p_2} = (2591 + 1) / 2 = 1296$. Tenemos por
    tanto las siguentes tablas de iteraciones para ambos números $p_1$ y $p_2$ respectivamente: \\

    % \begin{center}
        $\sqrt{4547}$
        \begin{tabular}{| c | c | c | c |}
            \hline Paso & $a$ & $a^2 + n$ & cociente \\ \hline
            1 & 1296 & 1682207 & 648 \\
            2 & 648 & 422495 & 325 \\
            3 & 325 & 108216 & 166 \\
            4 & 166 & 30147 & 90 \\
            5 & 90 & 10691 & 59 \\
            6 & 59 & 6072 & 51 \\
            7 & 51 & 5192 & 50 \\
            8 & 50 & 5091 & 50 \\ \hline
        \end{tabular}
    % \end{center}
        $\sqrt{2591}$
    % \begin{center}
        \begin{tabular}{| c | c | c | c |}
            \hline Paso & $a$ & $a^2 + n$ & cociente \\ \hline
            1 & 2274 & 5175623 & 1137 \\
            2 & 1137 & 1297316 & 570 \\
            3 & 570 & 329447 & 288 \\
            4 & 288 & 87491 & 151 \\
            5 & 151 & 27348 & 90 \\
            6 & 90 & 12647 & 70 \\
            7 & 70 & 9447 & 67 \\
            8 & 67 & 9036 & 67 \\ \hline
        \end{tabular} \\
    % \end{center}

    Con lo que hemos obtenido, en la última iteración de cada tabla, las respectivas partes enteras de la raíz cuadrada de 
    ambos primos, siendo para $p_1$ el valor 50 y para $p_2$ el 67.

    \newpage
    \textbf{Apartado III. \textit{Calcula las FCS de $\sqrt{p_1}$ y $\sqrt{p_2}$ aplicando el algoritmo que usa aritmética entera.}}

    La fracción continua simple de $\sqrt{4547}$ es la siguiente:
    \{58\{67 \{2, 3, 6, 1, 4, 3, 11, 1, 18, 2, 1, 7, 3, 1, 5, 9, 2, 5, 1, 1, 1, 9, 1, 2, 1, 1, 1, 4, 67, 4, 1, 1, 1, 2, 1, 9, 1, 1, 1, 5, 
    2, 9, 5, 1, 3, 7, 1, 2, 18, 1, 11, 3, 4, 1, 6, 3, 2, 134\}\}\} \\
    La cual podemos ver que su período tiene una longitud de 58.

    La fracción continua simple de $\sqrt{2591}$ es la siguiente:
    \{48 \{50 \{1, 9, 5, 3, 1, 7, 14, 2, 2, 2, 2, 1, 6, 1, 1, 3, 2, 1, 1, 1, 2, 19, 1, 49, 1, 19, 2, 1, 1, 1, 2, 3, 1, 1, 6, 1, 2, 2, 2, 
    2, 14, 7, 1, 3, 5, 9, 1, 100\}\}\}
    La cual podemos ver que su período tiene una longitud de 48.


\end{document}