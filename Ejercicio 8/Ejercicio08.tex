\documentclass[fleqn]{article}

%\pgfplotsset{compat=1.17}

\usepackage{mathexam}
\usepackage{amsmath}
\usepackage{amsfonts}
\usepackage{graphicx}
\usepackage{systeme}
\usepackage{microtype}
\usepackage{multirow}
\usepackage{pgfplots}
\usepackage{listings}
\usepackage{tikz}
\usepackage{dsfont} %Numeros reales, naturales...
\usepackage{cancel}
\usepackage{babel}

%\graphicspath{{images, }}
\newcommand*{\QED}{\hfill\ensuremath{\square}}

%Estructura de ecuaciones
\setlength{\textwidth}{15cm} \setlength{\oddsidemargin}{5mm}
\setlength{\textheight}{23cm} \setlength{\topmargin}{-1cm}



\author{David García Curbelo}
\title{Teoría de Números y Criptografía}

\pagestyle{empty}


\def\R{\mathds{R}}
\def\Z{\mathds{Z}}
\def\N{\mathds{N}}
\def\Q{\mathds{Q}}

\def\sup{$^2$}

\def\next{\quad \Rightarrow \quad}


\begin{document}
    \begin{center}
        \LARGE{\textbf{Ejercicio 7}} \\
        \Large{David García Curbelo} \\
    \end{center}

    \vspace{1cm}
    Toma tu número $n=191871308917122834687961459636870046909$ de la lista publicada para este ejercicio.


    \textbf{Apartado I. \textit{Pasa algunos test de primalidad para ver si $n$ es compuesto}}\\

    
    \newpage
    \textbf{Apartado II. \textit{En caso que tu $n$ sea probable primo, factoriza $n+1$ encontrando certificados de primalidad 
            para factores mayores de 10000}}\\

    \newpage
    \textbf{Apartado III. \textit{Con $P=1$, encuentra $Q$ natural mayor o igual que 2, tal que definan una sucesión de
            Lucas que certifique la primalidad $n$.}} \\


\end{document}