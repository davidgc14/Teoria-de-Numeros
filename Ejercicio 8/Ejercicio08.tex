\documentclass[fleqn]{article}

%\pgfplotsset{compat=1.17}

\usepackage{mathexam}
\usepackage{amsmath}
\usepackage{amsfonts}
\usepackage{graphicx}
\usepackage{systeme}
\usepackage{microtype}
\usepackage{multirow}
\usepackage{pgfplots}
\usepackage{listings}
\usepackage{tikz}
\usepackage{dsfont} %Numeros reales, naturales...
\usepackage{cancel}
\usepackage{babel}

%\graphicspath{{images, }}
\newcommand*{\QED}{\hfill\ensuremath{\square}}

%Estructura de ecuaciones
\setlength{\textwidth}{15cm} \setlength{\oddsidemargin}{5mm}
\setlength{\textheight}{23cm} \setlength{\topmargin}{-1cm}



\author{David García Curbelo}
\title{Teoría de Números y Criptografía}

\pagestyle{empty}


\def\R{\mathds{R}}
\def\Z{\mathds{Z}}
\def\N{\mathds{N}}
\def\Q{\mathds{Q}}

\def\sup{$^2$}

\def\next{\quad \Rightarrow \quad}


\begin{document}
    \begin{center}
        \LARGE{\textbf{Ejercicio 7}} \\
        \Large{David García Curbelo} \\
    \end{center}

    \vspace{1cm}
    Toma tu número $n=191871308917122834687961459636870046909$ de la lista publicada para este ejercicio.


    \textbf{Apartado I. \textit{Pasa algunos test de primalidad para ver si $n$ es compuesto}}\\
    Veamos los resultados de pasar el test de Fermat para las las bases 2, 3, 5 y 7. Usando el algoritmo de exponenciación rápida, tenemos
    los siguientes resultados de evaluar $a^{n-1} \pmod{n}$ para las bases mencionadas:

    \begin{center}
        Base 2
        \begin{tabular}{| c | c | c |} \hline
            Iteración & Exponente & Acumulado \\ \hline
            1 & 1 & 2 \\
            2 & 2 & 4 \\
            3 & 4 & 16 \\
            4 & 9 & 512 \\
            5 & 18 & 262144 \\
            6 & 36 & 68719476736 \\
            7 & 72 & 4722366482869645213696 \\
            8 & 144 & 118577020187434258021703433866563888073 \\
            9 & 288 & 29340227568084248416501045335641728100 \\
            $\cdots$ & $\cdots$ & $\cdots$ \\
            122 & 2997989201830044291999397806826094482 & 101675255918338275426907079426403493615 \\
            123 & 5995978403660088583998795613652188965 & 74303581080468529483843167643474362336 \\
            124 & 11991956807320177167997591227304377931 & 100644730683039452944647609244075852789 \\
            125 & 23983913614640354335995182454608755863 & 59117478806542687318662872566695394313 \\
            126 & 47967827229280708671990364909217511727 & 118234957613085374637325745133390788625 \\
            127 & 95935654458561417343980729818435023454 & -1  \\
            128 & 191871308917122834687961459636870046908 & 1 \\ \hline
        \end{tabular} \\
        ---

        Base 3
        \begin{tabular}{| c | c | c |} \hline
            Iteración & Exponente & Acumulado \\ \hline
            1 & 1 & 3\\
            2 & 2 & 9\\
            3 & 4 & 81\\
            4 & 9 & 19683\\
            5 & 18 & 387420489\\
            6 & 36 & 150094635296999121\\
            7 & 72 & 22528399544939174411840147874772641\\
            8 & 144 & 79863541316174948313858493560647916069\\
            9 & 288 & 21923773134371130434670971852591589643\\
            $\cdots$ & $\cdots$ & $\cdots$ \\
            122 & 2997989201830044291999397806826094482 & 164724765414792627514895963012658408736\\
            123 & 5995978403660088583998795613652188965 & 50895079125131643876702492628639013989\\
            124 & 11991956807320177167997591227304377931 & 8495266008180564491497240622868865667\\
            125 & 23983913614640354335995182454608755863 & 175843235986790788741521745667721490030\\
            126 & 47967827229280708671990364909217511727 & 118234957613085374637325745133390788625\\
            127 & 95935654458561417343980729818435023454 & -1 \\
            128 & 191871308917122834687961459636870046908 & 1\\ \hline
        \end{tabular} \\
        ---
        
        Base 5
        \begin{tabular}{| c | c | c |} \hline
            Iteración & Exponente & Acumulado \\ \hline
            1 & 1 & 5 \\
            2 & 2 & 25 \\
            3 & 4 & 625 \\
            4 & 9 & 1953125 \\
            5 & 18 & 3814697265625 \\
            6 & 36 & 14551915228366851806640625 \\
            7 & 72 & 109995655051602194850629945210045793959 \\
            8 & 144 & 72055542060367415663075562708228631321 \\
            9 & 288 & 16059290738605625158384538873481518775 \\
            $\cdots$ & $\cdots$ & $\cdots$ \\
            122 & 2997989201830044291999397806826094482 & 83073020954742175186051287482142023167 \\
            123 & 5995978403660088583998795613652188965 & 59165329114860589351210691706810981474 \\
            124 & 11991956807320177167997591227304377931 & 87509861727301316206372170982747614537 \\
            125 & 23983913614640354335995182454608755863 & 119817173846143518687612360340700291480 \\
            126 & 47967827229280708671990364909217511727 & 1 \\
            127 & 95935654458561417343980729818435023454 & 1 \\
            128 & 191871308917122834687961459636870046908 & 1 \\ \hline        
        \end{tabular}
        ---
        
        Base 7
        \begin{tabular}{| c | c | c |} \hline
            Iteración & Exponente & Acumulado \\ \hline
            1 & 1 & 7 \\
            2 & 2 & 49 \\
            3 & 4 & 2401 \\
            4 & 9 & 40353607 \\
            5 & 18 & 1628413597910449 \\
            6 & 36 & 2651730845859653471779023381601 \\
            7 & 72 & 93896753671824022897665880086441711349 \\
            8 & 144 & 139184725522475144930465216926335590537 \\
            9 & 288 & 88353904871015366688933124559184247632 \\
            $\cdots$ & $\cdots$ & $\cdots$ \\
            122 & 2997989201830044291999397806826094482 & 143919716633746330712290624655863621828 \\
            123 & 5995978403660088583998795613652188965 & 48370016607229836518498432319658144500 \\
            124 & 11991956807320177167997591227304377931 & 28480560223009031088627591725543683721 \\
            125 & 23983913614640354335995182454608755863 & 178621585979619985521227455109380454469 \\
            126 & 47967827229280708671990364909217511727 & -1 \\
            127 & 95935654458561417343980729818435023454 & 1 \\
            128 & 191871308917122834687961459636870046908 & 1 \\ \hline        
        \end{tabular}
    \end{center}

    Como hemos visto, para cada una de las bases se cumple que $a^{n-1} \equiv 1 \pmod{n}$, luego para
    $a = 2,3,5,7$ tenemos que $n$ es un posible primo de Fermat para dichas bases. Comprobemos ahora si nuestro número $n$ 
    pasa el test de Euler. Para ello, calculamos el símbolo de Jacobi $\left(\frac{a}{m}\right)$ para cada una de las bases 
    y comprobamos que coincida con el valor de $a^{(m-1)/2} \pmod{m}$, que es precisamente la penúltima iteración del 
    algoritmo realizado en el apartado anterior.\\
    \begin{enumerate}
        \item[$\bullet$] $\left(\frac{2}{m}\right) = (-1)^{(n^2 - 1) / 2} = -1$ por ser $n \equiv -3 \pmod{8}$.
        \item[$\bullet$] $\left(\frac{3}{m}\right) = \left(\frac{m}{3}\right) = \left(\frac{2}{3}\right) = -\left(\frac{1}{3}\right) = -1$.
        \item[$\bullet$] $\left(\frac{5}{m}\right) = \left(\frac{m}{5}\right) = \left(\frac{4}{5}\right) = - \left(\frac{2}{5}\right) = \left(\frac{1}{5}\right) = 1$.
        \item[$\bullet$] $\left(\frac{7}{m}\right) = \left(\frac{m}{7}\right) = \left(\frac{2}{7}\right) = - \left(\frac{1}{7}\right) = 1$.
    \end{enumerate}

    Vemos que dichos símbolos coinciden con la penúltima iteración del algoritmo de exponenciación rápida, luego $n$ ha pasado el test de Solovay-Strassen
    para las bases 2,3,5y 7, luego tenemos una probabilidad de primalidad del $1-\frac{1}{2^4} = 0.984375$.



    \newpage
    \textbf{Apartado II. \textit{En caso que tu $n$ sea probable primo, factoriza $n+1$ encontrando certificados de primalidad 
            para factores mayores de 10000}}\\
    Por el apartado anterior, tenemos altas probabilidades de que nuestro numero $n$ sea primo. Por ello procedemos a factorizar $n+1$.

    $m = n+1 = 191871308917122834687961459636870046910 = 2 \cdot 3^2 \cdot 5 \cdot 2131903432412475940977349551520778299$. Desconocemos si 
    $m_1 = 2131903432412475940977349551520778299$ es primo, pero es fácil ver que $2^{m_1 - 1} \not\equiv 1 \pmod{m_1}$ por lo que $m_1$ no 
    es primo. Así, aplicando ahora el método $\rho$ de Polard, obtenemos
    \begin{center}
        \begin{tabular}{| c | c | c | c |}
            \hline Paso & $x$ & $y$ & mcd \\ \hline
            1 & 2 & 5 & 1 \\
            2 & 5 & 677 & 1 \\
            3 & 26 & 210066388901 & 1 \\
            4 & 677 & 1895334587094284184613091101280776558 & 1 \\
            5 & 458330 & 78215585125484868093905659043889560 & 1 \\
            6 & 210066388901 & 1544705283024627326323128430540469400 & 1 \\
            7 & 44127887745906175987802 & 1886113287013530250529305226348579810 & 1 \\
            & $\cdots$ & $\cdots$ & 1 \\
            199 & 1870715726329717217294060583935591760 & 1060801355651140048392732542642863088 & 1 \\
            200 & 424068958678740670085019879061666598 & 198391725553609196458053623579235637 & 1 \\
            201 & 1144313711092545399133664641920726969 & 1087000882140671190941176670812395039 & 1 \\
            202 & 487249409427576944821049604574813716 & 1380997187252811280112785131151297400 & 1 \\
            203 & 1428917232945032853447041072009195027 & 557963052235651406262131922066187695 & 1 \\
            204 & 662668825124170441837361214480548672 & 454271395768955732082738715776001264 & 1 \\
            205 & 896153999982941990933718819692803994 & 1849723980397089800013259742497928909 & 154493 \\ \hline
        \end{tabular}
    \end{center}
    Con lo que hemos obtenido un factor de $m_1 = 154493 \cdot 13799352931281520463563718430743$. Para ver si son posibles primos, aplicamos el test de 
    Fermat con el que obtenemos que $m_2 = 154493$ pasa el test cumpliendo $a^{(m_2 - 1)} \equiv 1 \pmod{m_2}$ para las bases $a = 2,3,5,7$, con lo que tenemos 
    que es posible primo. Procedemos a buscar un certificado de primalidad mediante el algoritmo de Lucas-Lehmer, factorizando $m_2 - 1 = 154492 = 2^2 \cdot 38623$.
    Veamos si $m_{2,1} = 38623$ es primo. Para ello, aplicamos el test de Fermat y obtenemos que $2^{(m_2 - 1)} \not\equiv 1 \pmod{m_2}$, luego tenemos certificado de
    composición. Procedemos a obtener sus factores mediante el algoritmo $\rho$ de Polard:  
    \begin{center}
        \begin{tabular}{| c | c | c | c |}
            \hline Paso & $x$ & $y$ & mcd \\ \hline
            1 & 2 & 5 & 1 \\
            2 & 5 & 677 & 1 \\
            3 & 26 & 24562 & 1 \\
            4 & 677 & 33242 & 13 \\ \hline
        \end{tabular}
    \end{center}

    Tenemos así que $m_{2,1} = 38623 = 13 \cdot 2971$ (ambos factores son primos menores de 10000) luego tenemos completamente factorizado 
    $m_2 - 1 = 154492 = 2^2 \cdot 13 \cdot 2971$ y por tanto estamos en condiciones de encontrar un elemento primitivo:
    \begin{enumerate}
        \item[$\bullet$] $2^{(m_2 - 1)} \equiv 1 \pmod{m_2}$
        \item[$\bullet$] $2^{(m_2 - 1)/2} \not\equiv 1 \pmod{m_2}$    
        \item[$\bullet$] $2^{(m_2 - 1)/13} \not\equiv 1 \pmod{m_2}$
        \item[$\bullet$] $2^{(m_2 - 1)/2971} \not\equiv 1 \pmod{m_2}$
    \end{enumerate}
    Hemos obtenido un elemento primitivo, y por tanto un certificado de primalidad de 154493. Procedemos a estudiar la primalidad de 
    $m_3 = 13799352931281520463563718430743$. Para ello, aplicamos el test de Fermat y obtenemos que $2^{(m_3 - 1)} \not\equiv 1 \pmod{m_3}$, luego tenemos
    un certificado de composición. Aplicamos por tanto el algoritmo $\rho$ de Polard para encontrar sus factores, obteniendo
    una factorización de $m_3 = 5766560731 \cdot 2392995335520991752053$. 
    
    Veamos si cada uno de los factores es primo o no. Para ello, aplicamos el test de
    Fermat y obtenemos que para ambos candidatos $m_{3,1} = 5766560731$ y $m_{3,2} = 2392995335520991752053$, ambos pasan el test de Fermat $a^{(m_{3,i} - 1)} \equiv 1 \pmod{m_{3,i}}$
    para las bases $a = 2,3,5,7$ y con $i = 1,2$, con lo que tenemos que son posibles primos. Procedemos a buscar un certificado de primalidad mediante el algoritmo de Lucas-Lehmer.
    Factoricemos primero para este fin $m_{3,1} - 1 = 2 \cdot 3^2 \cdot 5 \cdot 7 \cdot 9153271$. Veamos si $m_{3,1,1} = 9153271$ es primo. Para ello, aplicamos el test de Fermat y
    obtenemos que $2^{(m_{3,1,1} - 1)} \not\equiv 1 \pmod{m_{3,1,1}}$, luego tenemos un certificado de composición. Procedemos a obtener sus factores mediante el algoritmo
    $\rho$ de Polard:
    \begin{center}
        \begin{tabular}{| c | c | c | c |}
            \hline Paso & $x$ & $y$ & mcd \\ \hline
            1 & 2 & 5 & 1 \\
            2 & 5 & 677 & 1 \\
            3 & 26 & 7972722 & 1 \\
            4 & 677 & 1349861 & 1 \\
            5 & 458330 & 4030875 & 1 \\
            6 & 7972722 & 9121035 & 1 \\
            7 & 592400 & 2030879 & 1 \\
            8 & 1349861 & 118386 & 1 \\
            9 & 1367894 & 3373269 & 1 \\
            10 & 4030875 & 2331996 & 127 \\ \hline
        \end{tabular}
    \end{center}
    Obtenemos una factorización de $m_{3,1,1} = 9153271 = 127 \cdot 72073$. Veamos si $m_{3,1,2} = 72073$ es primo o no. Vemos que dicho número pasa el test de Fermat para las bases 
    $a = 2,3,5,7$ y con $a^{(72073 - 1)} \equiv 1 \pmod{72073}$, con lo que tenemos que es posible primo. Procedemos a buscar un certificado de primalidad mediante el algoritmo
    de Lucas-Lehmer. Factoricemos por tanto $m_{3,1,2} - 1 = 2^3 \cdot 3^2 \cdot 1001 = 2^3 \cdot 3^2 \cdot 7 \cdot 11 \cdot 13 $. Ya tenemos factorizado completamente 
    $m_{3,1,2} - 1$, luego estamos en condiciones de buscar un elemento primitivo:
    \begin{enumerate}
        \item[$\bullet$] $5^{(m_{3,1,2} - 1)} \equiv 1 \pmod{m_{3,1,2}}$
        \item[$\bullet$] $5^{(m_{3,1,2} - 1)/2} \not\equiv 1 \pmod{m_{3,1,2}}$    
        \item[$\bullet$] $5^{(m_{3,1,2} - 1)/3} \not\equiv 1 \pmod{m_{3,1,2}}$
        \item[$\bullet$] $5^{(m_{3,1,2} - 1)/7} \not\equiv 1 \pmod{m_{3,1,2}}$
        \item[$\bullet$] $5^{(m_{3,1,2} - 1)/11} \not\equiv 1 \pmod{m_{3,1,2}}$
        \item[$\bullet$] $5^{(m_{3,1,2} - 1)/13} \not\equiv 1 \pmod{m_{3,1,2}}$
    \end{enumerate}
    Hemos obtenido un elemento primitivo, y por tanto un certificado de primalidad de $m_{3,1,2} = 72073$, y así un factorización completa de 
    $m_{3,1} - 1 = 2 \cdot 3^2 \cdot 5 \cdot 7 \cdot 127 \cdot 72073$, y por tanto estamos en condiciones de encontrar un elemento primitivo para $m_{3,1} = 5766560731$:
    \begin{enumerate}
        \item[$\bullet$] $2^{(m_{3,1} - 1)} \equiv 1 \pmod{m_{3,1}}$
        \item[$\bullet$] $2^{(m_{3,1} - 1)/2} \not\equiv 1 \pmod{m_{3,1}}$    
        \item[$\bullet$] $2^{(m_{3,1} - 1)/3} \not\equiv 1 \pmod{m_{3,1}}$
        \item[$\bullet$] $2^{(m_{3,1} - 1)/5} \not\equiv 1 \pmod{m_{3,1}}$
        \item[$\bullet$] $2^{(m_{3,1} - 1)/7} \not\equiv 1 \pmod{m_{3,1}}$
        \item[$\bullet$] $2^{(m_{3,1} - 1)/127} \not\equiv 1 \pmod{m_{3,1}}$
        \item[$\bullet$] $2^{(m_{3,1} - 1)/72073} \not\equiv 1 \pmod{m_{3,1}}$
    \end{enumerate}
    Hemos obtenido un elemento primitivo, y por tanto un certificado de primalidad de $m_{3,1} = 5766560731$. 
    
    Nos falta por estudiar la primalidad de 
    $m_{3,2} = 2392995335520991752053$. Para ello, encontremos una factorización completa de $m_{3,2} -1 = 2^2 \cdot 598248833880247938013$ y falta ver si
    $m_{3,2,1} = 598248833880247938013$ es primo. Pero vemos que pasa el Test de Fermat para las bases $a = 2,3,5,7$,
    con lo que tenemos que es posible primo. Procedemos a buscar un certificado de primalidad mediante el algoritmo de Lucas-Lehmer, factorizando 
    $m_{3,2,1} -1 = 2^2 \cdot 3 \cdot 49854069490020661501$. Vemos que, aplicando el test de Fermat, $2^{(m_{3,2,2} - 1)} \not\equiv 1 \pmod{m_{3,2,2}}$, luego tenemos 
    certificado de composición. Aplicando el algoritmo $\rho$ de Polard:
    \begin{center}
        \begin{tabular}{| c | c | c | c |}
            \hline Paso & $x$ & $y$ & mcd \\ \hline
            1 & 2 & 5 & 1 \\
            2 & 5 & 677 & 1 \\
            3 & 26 & 210066388901 & 1 \\
            4 & 677 & 48154026845582945885 & 1 \\
            5 & 458330 & 16477550820312657009 & 1 \\
            6 & 210066388901 & 8767016751718741827 & 17 \\ \hline
        \end{tabular}
    \end{center}
    Obtenemos así un factor $m_{3,2,2} = 49854069490020661501 = 17 \cdot 2932592322942391853$. Repetimos el proceso, y vemos que aplicando el test de
    Fermat, $2^{(m_{3,2,3} - 1)} \not\equiv 1 \pmod{m_{3,2,3}}$, con $m_{3,2,3}$, luego tenemos certificado de composición. Aplicando el algoritmo $\rho$ de Polard:
    \begin{center}
        \begin{tabular}{| c | c | c | c |}
            \hline Paso & $x$ & $y$ & mcd \\ \hline
            1 & 2 & 5 & 1 \\
            2 & 5 & 677 & 1 \\
            3 & 26 & 210066388901 & 1 \\
            4 & 677 & 1232549678504676237 & 1 \\
            5 & 458330 & 1814589205600697744 & 1 \\
            6 & 210066388901 & 2901832105833958121 & 1 \\
            7 & 1171062592005775711 & 433343035452068346 & 1 \\
            8 & 1232549678504676237 & 530026126697573956 & 1 \\
            & $\cdots$ & $\cdots$ & 1 \\
            105 & 1072907927239829481 & 633060779998309531 & 1 \\
            106 & 1473715790323534215 & 1476795736927761280 & 1 \\
            107 & 700874798274490282 & 1921604684212310878 & 1 \\
            108 & 1631026691484340616 & 2602941659810613916 & 1 \\
            109 & 2098486954844577194 & 106004085321239324 & 1 \\
            110 & 910283951658615124 & 2174617737679806385 & 1 \\
            111 & 2326822244406153002 & 1356215947241368806 & 1 \\
            112 & 2780900295665156451 & 2177037073141657912 & 8389 \\ \hline
        \end{tabular}
    \end{center}
    Obtenemos así otro factor $m_{3,2,1} -1 = 2^2 \cdot 3 \cdot 17 \cdot 8389 \cdot 349575911663177$. Repetimos el proceso, y vemos que para $m_{3,2,3} = 349575911663177$
    aplicando el test de Fermat, $2^{(m_{3,2,3} - 1)} \not\equiv 1 \pmod{m_{3,2,3}}$, con $m_{3,2,3}$, luego tenemos certificado de composición. Aplicando el algoritmo $\rho$ de Polard
    obtenemos:
    \begin{center}
        \begin{tabular}{| c | c | c | c |}
            \hline Paso & $x$ & $y$ & mcd \\ \hline
            1 & 2 & 5 & 1 \\
            2 & 5 & 677 & 1 \\
            3 & 26 & 210066388901 & 1 \\
            4 & 677 & 294589891977312 & 1 \\
            5 & 458330 & 290224068809114 & 1 \\
            6 & 210066388901 & 2463117925844 & 1 \\
            7 & 332863845795938 & 218480901392043 & 1 \\
            & $\cdots$ & $\cdots$ & 1 \\
            2074 & 90963594969107 & 304227142127437 & 1 \\
            2075 & 48834153381626 & 159526109489353 & 1 \\
            2076 & 134463322576653 & 28163583826849 & 1 \\
            2077 & 245280188018629 & 284685468917185 & 1 \\
            2078 & 42109524707488 & 227174816698981 & 1 \\
            2079 & 338951848011268 & 261789712391706 & 1 \\
            2080 & 257496113186742 & 185895937987750 & 2291797 \\ \hline
        \end{tabular}
    \end{center}
    Obtenemos asi otro factor $m_{3,2,1} -1 = 2^2 \cdot 3 \cdot 17 \cdot 8389 \cdot 2291797 \cdot 152533541$. Veamos si estos dos últimos factores son primos.
    Vemos que ambos pasan el test de Fermat para las bases $a = 2,3,5,7$, con lo que tenemos altas probabilidades de primalidad. Aplicamos por tanto a ambos
    el test de Lucas-Lehmer, y buscamos una factorización de 
    $m_{3,2,1,1} -1 = 2291796 = 2^2 \cdot 3^2 \cdot 13 \cdot 59 \cdot 83$ y de 
    $m_{3,2,1,2} -1 = 152533540 = 2^2 \cdot 5 \cdot 67 \cdot 89 \cdot 1279$ (los tres últimos factores en ambos números han sido calculados mediante el algoritmo 
    $rho$ de Polard). Así, estamos en condiciones de buscar un elemento primitivo para cada uno de los candidatos a primo $m_{3,2,1,1} = 2291797$ y $m_{3,2,1,2} = 152533541$:
    \begin{enumerate}
        \item[$\bullet$] $2^{(m_{3,2,1,1} - 1)} \equiv 1 \pmod{m_{3,2,1,1}}$
        \item[$\bullet$] $2^{(m_{3,2,1,1} - 1)/2} \not\equiv 1 \pmod{m_{3,2,1,1}}$    
        \item[$\bullet$] $2^{(m_{3,2,1,1} - 1)/3} \not\equiv 1 \pmod{m_{3,2,1,1}}$
        \item[$\bullet$] $2^{(m_{3,2,1,1} - 1)/13} \not\equiv 1 \pmod{m_{3,2,1,1}}$
        \item[$\bullet$] $2^{(m_{3,2,1,1} - 1)/59} \not\equiv 1 \pmod{m_{3,2,1,1}}$
        \item[$\bullet$] $2^{(m_{3,2,1,1} - 1)/83} \not\equiv 1 \pmod{m_{3,2,1,1}}$
    \end{enumerate}
    ---
    \begin{enumerate}
        \item[$\bullet$] $3^{(m_{3,2,1,2} - 1)} \equiv 1 \pmod{m_{3,2,1,2}}$
        \item[$\bullet$] $3^{(m_{3,2,1,2} - 1)/2} \not\equiv 1 \pmod{m_{3,2,1,2}}$    
        \item[$\bullet$] $3^{(m_{3,2,1,2} - 1)/5} \not\equiv 1 \pmod{m_{3,2,1,2}}$
        \item[$\bullet$] $3^{(m_{3,2,1,2} - 1)/67} \not\equiv 1 \pmod{m_{3,2,1,2}}$
        \item[$\bullet$] $3^{(m_{3,2,1,2} - 1)/89} \not\equiv 1 \pmod{m_{3,2,1,2}}$
        \item[$\bullet$] $3^{(m_{3,2,1,2} - 1)/1279} \not\equiv 1 \pmod{m_{3,2,1,2}}$
    \end{enumerate}
    Hemos encontrado un elemento primitivo para cada candidato a primo, luego tenemos certificado de primalidad de ambos. Ahora, tenemos factorizado en primos 
    $m_{3,2,1} -1 = 2^2 \cdot 3 \cdot 17 \cdot 8389 \cdot 2291797 \cdot 152533541$, y estamos en condiciones de encontrar un elemento primitivo para 
    $m_{3,2,1} = 598248833880247938013$:
    \begin{enumerate}
        \item[$\bullet$] $2^{(m_{3,2,1} - 1)} \equiv 1 \pmod{m_{3,2,1}}$
        \item[$\bullet$] $2^{(m_{3,2,1} - 1)/2} \not\equiv 1 \pmod{m_{3,2,1}}$    
        \item[$\bullet$] $2^{(m_{3,2,1} - 1)/3} \not\equiv 1 \pmod{m_{3,2,1}}$
        \item[$\bullet$] $2^{(m_{3,2,1} - 1)/17} \not\equiv 1 \pmod{m_{3,2,1}}$
        \item[$\bullet$] $2^{(m_{3,2,1} - 1)/8389} \not\equiv 1 \pmod{m_{3,2,1}}$
        \item[$\bullet$] $2^{(m_{3,2,1} - 1)/2291797} \not\equiv 1 \pmod{m_{3,2,1}}$
        \item[$\bullet$] $2^{(m_{3,2,1} - 1)/152533541} \not\equiv 1 \pmod{m_{3,2,1}}$
    \end{enumerate}
    Hemos encontrado un elemento primitivo para $m_{3,2,1} = 598248833880247938013$, luego tenemos certificado de primalidad. Ahora tenemos factorizado en primos
    $m_{3,2} - 1 = 2^2 \cdot 598248833880247938013$, y estamos en condiciones de encontrar un elemento primitivo para $m_{3,2} = 2392995335520991752053$:
    \begin{enumerate}
        \item[$\bullet$] $2^{(m_{3,2,1} - 1)} \equiv 1 \pmod{m_{3,2,1}}$
        \item[$\bullet$] $2^{(m_{3,2,1} - 1)/2} \not\equiv 1 \pmod{m_{3,2,1}}$    
        \item[$\bullet$] $2^{(m_{3,2,1} - 1)/598248833880247938013} \not\equiv 1 \pmod{m_{3,2,1}}$
    \end{enumerate}
    Hemos encontrado un elemento primitivo para $m_{3,2} = 2392995335520991752053$, luego tenemos certificado de primalidad. 

    Con esto ya hemos terminado, pues hemos encontrado una factorización en primos del número pedido 
    $n+1 = 2 \cdot 3^2 \cdot 5 \cdot 154493 \cdot 5766560731 \cdot 2392995335520991752053$

    




    \newpage
    \textbf{Apartado III. \textit{Con $P=1$, encuentra $Q$ natural mayor o igual que 2, tal que definan una sucesión de
            Lucas que certifique la primalidad $n$.}} \\


\end{document}