\documentclass[fleqn]{article}

%\pgfplotsset{compat=1.17}

\usepackage{mathexam}
\usepackage{amsmath}
\usepackage{amsfonts}
\usepackage{graphicx}
\usepackage{systeme}
\usepackage{microtype}
\usepackage{multirow}
\usepackage{pgfplots}
\usepackage{listings}
\usepackage{tikz}
\usepackage{dsfont} %Numeros reales, naturales...
\usepackage{cancel}
\usepackage{babel}

%\graphicspath{{images, }}
\newcommand*{\QED}{\hfill\ensuremath{\square}}

%Estructura de ecuaciones
\setlength{\textwidth}{15cm} \setlength{\oddsidemargin}{5mm}
\setlength{\textheight}{23cm} \setlength{\topmargin}{-1cm}



\author{David García Curbelo}
\title{Teoría de Números y Criptografía}

\pagestyle{empty}


\def\R{\mathds{R}}
\def\Z{\mathds{Z}}
\def\N{\mathds{N}}
\def\Q{\mathds{Q}}

\def\sup{$^2$}

\def\next{\quad \Rightarrow \quad}


\begin{document}
    \begin{center}
        \LARGE{\textbf{Ejercicio 7}} \\
        \Large{David García Curbelo} \\
    \end{center}

    \vspace{1cm}
    Toma tu número $n=4230659086792057869605292356791$ de la lista publicada para el ejercicio 3. Sea $d$ el 
    primer elemento de la sucesión $5, -7, 9, -11, 13, \dots$ que satisface que el símbolo de Jacobi es $(d|n) = -1$.


    \textbf{Apartado I. \textit{Con $P=1$, $Q = (1-n)/4$, define el e.c. $\alpha$ y sus sucesiones de Lucas asociadas.}} 


    \newpage
    \textbf{Apartado II. \textit{Si $n$ primo, ¿Qué debería pasarle a $V_r$, $U_r$, módulo $n$? ¿Y a $V_{r/2}$, $U_{r/2}$?
            Calcula los términos $V_r$, $U_r$, $V_{r/2}$, $U_{r/2}$ módulo $n$, de las sucesiones de Lucas.
            ¿Tu $n$ verifica el TPF para el entero cuadrático $\alpha$?}}

    
    \newpage
    \textbf{Apartado III. \textit{Factoriza $r = n+1$ y para cada factor primo $p$ suyo, calcula $U_{r/p}$.
            ¿Cuál es el rango de Lucas $w(n)$? ¿Qué deduces sobre la primalidad de tu $n$?}}
            



\end{document}