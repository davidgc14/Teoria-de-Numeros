\documentclass[fleqn]{article}

%\pgfplotsset{compat=1.17}

\usepackage{mathexam}
\usepackage{amsmath}
\usepackage{amsfonts}
\usepackage{graphicx}
\usepackage{systeme}
\usepackage{microtype}
\usepackage{multirow}
\usepackage{pgfplots}
\usepackage{listings}
\usepackage{tikz}
\usepackage{dsfont} %Numeros reales, naturales...
\usepackage{cancel}
\usepackage{babel}

%\graphicspath{{images, }}
\newcommand*{\QED}{\hfill\ensuremath{\square}}

%Estructura de ecuaciones
\setlength{\textwidth}{15cm} \setlength{\oddsidemargin}{5mm}
\setlength{\textheight}{23cm} \setlength{\topmargin}{-1cm}



\author{David García Curbelo}
\title{Teoría de Números y Criptografía}

\pagestyle{empty}


\def\R{\mathds{R}}
\def\Z{\mathds{Z}}
\def\N{\mathds{N}}
\def\Q{\mathds{Q}}

\def\sup{$^2$}

\def\next{\quad \Rightarrow \quad}


\begin{document}
    \begin{center}
        \LARGE{\textbf{Ejercicio 7}} \\
        \Large{David García Curbelo} \\
    \end{center}

    \vspace{1cm}
    Toma tu número $n=4230659086792057869605292356791$ de la lista publicada para el ejercicio 3. Sea $d$ el 
    primer elemento de la sucesión $5, -7, 9, -11, 13, \dots$ que satisface que el símbolo de Jacobi es $(d|n) = -1$.


    \textbf{Apartado I. \textit{Con $P=1$, $Q = (1-d)/4$, define el e.c. $\alpha$ y sus sucesiones de Lucas asociadas.}}\\
    Calculamos primero el valor de $d$ mediante el símbolo de Jacobi:

    \begin{enumerate}
        \item[-] $\left(\frac{5}{n}\right) = 1$
        \item[-] $\left(\frac{-7}{n}\right) = 1$
        \item[-] $\left(\frac{9}{n}\right) = 1$
        \item[-] $\left(\frac{-11}{n}\right) = 1$
        \item[-] $\left(\frac{13}{n}\right) = -1$   
    \end{enumerate}

    Hemos encontrado el valor de $d = 13$ que nos interesa. Así podemos hayar la forma explícita de $P = 1$ y $Q = -3$.
    De la misma manera podemos hayar la forma explícita de $\Delta = P^2 - 4Q = 13$ y por tanto obtener
    $\alpha = \frac{P + \sqrt{\Delta}}{2} = \frac{1 + \sqrt{13}}{2}$. \\
    Las sucesiones de lucas asociadas son las siguientes:
    \begin{enumerate}
        \item[$\bullet$] $V_n = P \cdot V_{n-1} - Q \cdot V_{n-2} = V_{n-1} + 3 \cdot V_{n-2}$ \\ 
        \item[$\bullet$] $U_n = P \cdot U_{n-1} - Q \cdot U_{n-2} = U_{n-1} + 3 \cdot U_{n-2}$ \\
    \end{enumerate}
    Con $V_0 = 2$, $V_1 = P$, $U_0 = 0$, $U_1 = 1$.

    \newpage
    \textbf{Apartado II. \textit{Si $n$ primo, ¿Qué debería pasarle a $V_r$, $U_r$, módulo $n$? ¿Y a $V_{r/2}$, $U_{r/2}$?
            Calcula los términos $V_r$, $U_r$, $V_{r/2}$, $U_{r/2}$ módulo $n$, de las sucesiones de Lucas.
            ¿Tu $n$ verifica el TPF para el entero cuadrático $\alpha$?}}\\
    Si tomamos $n$ suponiendo que es primo (y con $r = n+1$), por la tercera versión del TPF para elementos cuadráticos tenemos que, como 
    $\left(\frac{\Delta}{n}\right) = -1$ por definición, tienen que cumplirse las siguientes ecuaciones:\\ 
    $\left\{
    \begin{aligned}
        U_{n-\left(\frac{\Delta}{n}\right)} \equiv 0 \pmod{n} &\next U_{n+1} \equiv 0 \pmod{n} \\
        V_{n-\left(\frac{\Delta}{n}\right)} \equiv 2Q \pmod{n} &\next V_{n+1} \equiv -6 \pmod{n} \\
    \end{aligned}
    \right.$\\
    Ahora, para las consideraciones de los términos en la iteración $r/2$, consideremos las U-fórmulas binarias y la V-fórmula en función de U: \\
    $\left\{
        \begin{aligned}
            &U_{2k} = 2 U_k U_{k+1} - P U_k^2\\
            &U_{2k+1} = U_{k+1}^2 - Q U_k^2\\
            &U_{2k+2} = P U_{k+1}^2 - 2Q U_k U_{k+1} \\
            &V_k = 2 U_k U_{k+1} - P U_k\\
            &V_{2k} = V_{k}^2 - 2Q^k\\
        \end{aligned}
    \right.$\\
    Vemos que, por la primera y la penúltima fórmula, obtenemos $U_{2k} = V_k \cdot U_k$, y por la última tenemos $ V_{k}^2 = V_{2k} + 2Q^k$.
    Por tanto obtenemos las siguientes expresiones para los términos de $U$ y $V$ en la iteración $r/2$:\\
    $\left\{
        \begin{aligned}
            &U_{k} = V_{k/2} \cdot U_{k/2} \\
            &V_{k/2} = \sqrt{V_{k} + 2Q^{k/2}} \\
        \end{aligned}
    \right.$\\
    A las que aplicando restricción módulo $n$ obtenemos  $V_{r/2} \cdot U_{r/2} = U_{r} \equiv 0 \pmod{n}$, con lo que tenemos que $U_{r/2} \equiv 0 \pmod{n}$ 
    y por las fórmulas vistas $V_n \equiv -6 \pmod{n}$. Estas deducciones se pueden comprobar en la tabla de iteraciones de la página siguiente.
    
    Ahora calculamos $Q^{r/2}$, para el que usamos el algoritmo de la exponenciación rápida, donde $r/2 = 2115329543396028934802646178396$ (par), luego
    $(-3)^{\frac{r}{2}} = 3^{\frac{r}{2} } \equiv -3 \pmod{n}$. Además, por el punto anterior tenemos que $V_{n+1} \equiv -6 \pmod{n}$, luego tenemos
    $V_{r/2}^2 = V_{r} + 2Q^{r/2} \equiv -12 \pmod{n}$. Calculemos a continuación las iteraciones:\\
    \begin{center}
        \begin{tabular}{| c | c | c | c |} \hline
            Paso & $k$ & $U_k$ & $U_{k+1}$ \\ \hline
            0 & 0 & 0 & 1 \\
            1 & 1 & 1 & 1 \\
            2 & 3 & 4 & 7 \\
            3 & 6 & 40 & 97 \\
            4 & 13 & 14209 & 32689 \\
            5 & 26 & 727060321 & 1674257764 \\
            6 & 53 & 4388989191432148819 & 10106857384297773160 \\
            7 & 106 & 1206025542555073975697218196332 & 1191951767210935168446627691804 \\
            8 & 213 & 2224571724374272540640955371025 & 2781733333404799351094718353632 \\
            9 & 427 & 2724188379238101395143688701111 & 1040521496327043120557508055156 \\
            10 & 854 & 2240881567061812458486435864069 & 1121493515710978151287472213176 \\
            11 & 1708 & 2440177787821014965176523715987 & 281007983241315548690694467405 \\
            12 & 3417 & 693288824173917621822948187874 & 2371328698565907497147549144516 \\
            13 & 6834 & 3721998043825565111678776090720 & 2014292397858539868209996806393 \\
            14 & 13669 & 4205898394184130603133532495000 & 1211143114040918590331860328115 \\
            15 & 27339 & 384898754315453633874589689569 & 1982624428813654223520306030647 \\
            16 & 54679 & 2205114962351087054439123449795 & 3559492677168281359056314219403 \\
            17 & 109359 & 3056059869227629036204333889051 & 1423758336170861758017330367997 \\
            18 & 218719 & 3551423950121026880918215709294 & 3414073489724804099292817100277 \\
            19 & 437439 & 3460551003854904659688814626658 & 475591476867243768628155700978 \\
            20 & 874879 & 2460043190695490427945140061060 & 2247495142981071956725741493150 \\
            21 & 1749759 & 2166255830514324913128887404298 & 4112602938964843042151172594379 \\
            22 & 3499519 & 2608440951992315234653712945429 & 3239909279342061175967085769138 \\
            23 & 6999038 & 269920817574771798201888907821 & 1417378943408722380969677701581 \\
            24 & 13998076 & 1213435973533037509031355988281 & 3783608230937959308889385710867 \\
            25 & 27996153 & 652278378486732728206101158747 & 2826843898299710299380047780622 \\
            26 & 55992306 & 3884562517666216068130783262597 & 3181755662028242029634237058095 \\
            27 & 111984613 & 3247049328585303344489979652255 & 2695375132371359511468626840344 \\
            28 & 223969227 & 29779707979250577247359439087 & 2586696179189130169385172256763 \\
            29 & 447938454 & 2260479589319694550571971648344 & 28365850218704328636875632404 \\
            30 & 895876908 & 40965214621645240277380001145 & 453880624606167197749719810379 \\
            31 & 1791753817 & 510819107833845020980074302432 & 313598674112025178633287820140 \\
            32 & 3583507635 & 1541920834587226435934920973919 & 1281380269421394231209489702707 \\
            33 & 7167015270 & 953453786545491725568830749410 & 3264083616320700894968456708711 \\
            34 & 14334030540 & 2660475695792792413183733918485 & 682256232166213429514763646156 \\
            35 & 28668061080 & 2471751791161736882019886980384 & 657663229152715303675902651250 \\
            36 & 57336122161 & 4012606612333337626035394734648 & 2804823955982230319886947406902 \\
            37 & 114672244323 & 3105722382171480468348775381447 & 477563165797380354088428308632 \\
            38 & 229344488647 & 4067941019089584055244924617185 & 4212353841700088430611524932730 \\
            39 & 458688977294 & 3040575168173013589130366088093 & 3722956195285441553305565547109 \\
            40 & 917377954589 & 1678208648515403299696839198453 & 3705092999866615500274862218983 \\
            41 & 1834755909178 & 532263425061440016849618227413 & 357757549070238141197751942066 \\
            42 & 3669511818356 & 1628458654609289219083491675866 & 3267737650201680812221557869110 \\
            43 & 7339023636712 & 2401816628687239060277716038834 & 3880298180626118226409911223667 \\
            44 & 14678047273425 & 1356673018331891593458139273057 & 2890292439746692251495466896840 \\
            45 & 29356094546851 & 3998808044076379473204086068653 & 2927942947280424678606715894070 \\
            46 & 58712189093702 & 1002995773310192130560537046201 & 4075679818759325537695888936151 \\
            47 & 117424378187404 & 3638296379551707851196335475685 & 1977034590424034024427090686680 \\
            48 & 234848756374809 & 1199172975890223611843058638724 & 1520658918620734497202404032804 \\
            49 & 469697512749618 & 437752049599915428470389844843 & 1605082210629355310149354100408 \\
            50 & 939395025499236 & 2553932259872298234573825312571 & 3927868577358750705243248501250 \\
            51 & 1878790050998472 & 1782309769519364898725360825272 & 2526853358658154087876176798152 \\
            52 & 3757580101996944 & 3239951839968474674572277570082 & 1754660279628164450379321810440 \\ \hline
        \end{tabular}
    \end{center}
    \begin{center}
        \begin{tabular}{| c | c | c | c |} \hline
            Paso & $k$ & $U_k$ & $U_{k+1}$ \\ \hline
            53 & 7515160203993889 & 772554330115870989869155980205 & 2299312592280735131326723596533 \\
            54 & 15030320407987779 & 1161295187543971288033287164083 & 1980381983326016764815674111422 \\
            55 & 30060640815975559 & 3869095159199140697744395973369 & 1667415590678572485023488327919 \\
            56 & 60121281631951118 & 3689095507650274731217247011582 & 3006237437411835621207200203718 \\
            57 & 120242563263902237 & 361371593030257294943867729495 & 3698797394640497577013424035459 \\
            58 & 240485126527804474 & 4037023125584983642564393915119 & 283742619493790585931991704006 \\
            59 & 480970253055608949 & 609726694192642884902662201458 & 2010407164979210736305851260234 \\
            60 & 961940506111217898 & 2203795371142190303566191728217 & 767987694610083807997840429856 \\
            61 & 1923881012222435797 & 3092675007235659356534321074322 & 3459268692299825351158941463634 \\
            62 & 3847762024444871594 & 2102850815934169344305467567332 & 271108870520556852497276813415 \\
            63 & 7695524048889743188 & 911953416852421858603161607959 & 4179963076450375966867425305261 \\
            64 & 15391048097779486377 & 1186362651443553662947808482586 & 4045243062085195470686897804606 \\
            65 & 30782096195558972755 & 1069320115282748880522813385373 & 2465277257460566664141659917084 \\
            66 & 61564192391117945511 & 927177355366383141186254893116 & 1711223740035001107425162723873 \\
            67 & 123128384782235891022 & 2192899861152931969015541459362 & 2854255104631445313318084459842 \\
            68 & 246256769564471782045 & 729713940011399371026437205705 & 392083858637607540459658697976 \\
            69 & 492513539128943564091 & 2411782393514892043961867952206 & 1919593387624166151594118056998 \\
            70 & 985027078257887128182 & 2332145702477533882514640306496 & 1457480691950369735595535602847 \\
            71 & 1970054156515774256365 & 2979423589421281372928146645197 & 1710506592252300092854460323805 \\
            72 & 3940108313031548512731 & 863267501283303413041490436075 & 3598828613629915647475681835696 \\
            73 & 7880216626063097025463 & 428774538957186445089724910224 & 513882096213460020801917472892 \\
            74 & 15760433252126194050927 & 168200342518927822986564014638 & 264343055694671161506105581018 \\
            75 & 31520866504252388101855 & 3816996798390858806323434841261 & 3891147952995237099943079093383 \\
            76 & 63041733008504776203711 & 3785995490462982519501620025023 & 2448051585276976970554927717262 \\
            77 & 126083466017009552407422 & 2423988303408911371130583584366 & 1312066627441283175409769905070 \\
            78 & 252166932034019104814844 & 3944321132320519918692454869933 & 1479210302600363549365247348428 \\
            79 & 504333864068038209629689 & 2963024971525598142917209195324 & 1116768296098713161305969608172 \\
            80 & 1008667728136076419259379 & 4068903995066038179053611236566 & 3444752725337014154232687372550 \\
            81 & 2017335456272152838518758 & 1093531928391249018908754849400 & 1637639258829466856318313442665 \\
            82 & 4034670912544305677037517 & 4105475795626561399926924789347 & 2517746173429219115190407524569 \\
            83 & 8069341825088611354075035 & 3847141978989485121457722752996 & 4170357766961425894394556266816 \\
            84 & 16138683650177222708150071 & 2332334721165214836150765834138 & 2389478295031326272496489651079 \\
            85 & 32277367300354445416300143 & 2909550926215054970891480103336 & 470634721964005296753726917670 \\
            86 & 64554734600708890832600286 & 1687568250026306721350886995287 & 3680073732370416625026882542317 \\
            87 & 129109469201417781665200572 & 2088989998632477130755198486997 & 3458947781078041312110245245284 \\
            88 & 258218938402835563330401144 & 1592301268199309764611279620628 & 458204221306629199977148706924 \\
            89 & 516437876805671126660802289 & 392487535493950293919665024324 & 690660759799319557733225888240 \\
            90 & 1032875753611342253321604579 & 4071085333986147953587036809430 & 2352859312039741540036118108373 \\
            91 & 2065751507222684506643209158 & 3236598351193375409154738398645 & 493555714260808026523843813575 \\
            92 & 4131503014445369013286418317 & 1014013834014097741992720133965 & 1803871217829794019117781551089 \\
            93 & 8263006028890738026572836634 & 753760903567690576383788648926 & 4108274966803903777477949208925 \\
            94 & 16526012057781476053145673268 & 2248132779990784688478570879807 & 837827279477629125322250435241 \\
            95 & 33052024115562952106291346537 & 2661377540275603880511653852102 & 3115999644169315640252873472965 \\
            96 & 66104048231125904212582693074 & 1980109465271165993915830502869 & 322566446727654064099010975109 \\
            97 & 132208096462251808425165386149 & 2414153180482019947665784051432 & 1147944998430130699105537632557 \\
            98 & 264416192924503616850330772299 & 249567450660551621575252091327 & 1933555353481002084684640856244 \\
            99 & 528832385849007233700661544599 & 762432504487272216321147199828 & 381216252243636108160573599914 \\
            100 & 1057664771698014467401323089198 & 0 & 400308804229400273828849410268 \\
            101 & 2115329543396028934802646178396 & 0 & 1385230094848320756564637585261 \\
            102 & 4230659086792057869605292356792 & 0 & 4230659086792057869605292356788 \\ \hline
        \end{tabular}
    \end{center}
    
    Vemos que tenemos en las dos últimas iteraciones los valores de $U_r = U_{r/2} = 0$, además de tener $U_{r + 1} = 4230659086792057869605292356788$ y 
    $U_{r/2 + 1} = 1385230094848320756564637585261$. Por tanto, como $V_k = 2 U_{k+1} - P U_{k}$ tenemos que $V_r = 8461318173584115739210584713576 \equiv -6$ y
    $V_{r/2} = 2770460189696641513129275170522$, lo cual cumple con la condición vista antes $V_{r/2}^2 \equiv -12 \pmod{n}$.

    Como $V_r \equiv -6 \equiv 2Q \pmod{n}$ y $U_r \equiv 0 \pmod{n}$, tenemos que nuestro número $n$ cumple la tercera versión del TPF para el elemento cuadrático
    $\alpha = \frac{1 + \sqrt{13}}{2}$.
    
        



    \newpage
    \textbf{Apartado III. \textit{Factoriza $r = n+1$ y para cada factor primo $p$ suyo, calcula $U_{r/p}$.
            ¿Cuál es el rango de Lucas $w(n)$? ¿Qué deduces sobre la primalidad de tu $n$?}} \\
    Factorizamos $r = n+1 = 4230659086792057869605292356792 = 2^3 \cdot 528832385849007233700661544599$. Desconocemos si
    $m_1 = 528832385849007233700661544599$ es primo o no. Para saberlo aplicamos el test de composición de Fermat y obtenemos que 
    $2^{m_1 -1} \not\equiv 1 \pmod{m_1}$ (algoritmo de exponenciación rápida) con lo que obtenemos un certificado de composición. Ahora factorizando mediante el algoritmo $\rho$
    de Polard, obtenemos un divisor 4349 en 46 iteraciones, quedando por tanto $528832385849007233700661544599 = 4349 \cdot 121598617118649628351497251$.

    Veamos a continuación si $m_2 = 121598617118649628351497251$ es primo o no. Para saberlo aplicamos el test de composición de Fermat y
    obtenemos que $2^{m_2 -1} \not\equiv 1 \pmod{m_2}$ (algoritmo de exponenciación rápida) con lo que obtenemos un certificado de composición. 
    Ahora factorizando mediante el algoritmo $\rho$ de Polard, obtenemos un divisor 62347 en 46 iteraciones, quedando por tanto $121598617118649628351497251 = 62347 \cdot 1950352336417945183433$.
    
    Repetimos el mismo proceso para $m_3 = 1950352336417945183433$. Aplicando el test de composición de Fermat, obtenemos que 
    $2^{m_3 -1} \not\equiv 1 \pmod{m_3}$ (algoritmo de exponenciación rápida) con lo que obtenemos un certificado de composición.
    Ahora factorizando mediante el algoritmo $\rho$ de Polard, obtenemos un divisor 1924630699 en 26395 iteraciones, quedando por tanto 
    $1950352336417945183433 = 1924630699 \cdot 1013364453467$.
    
    Veamos por último si $m_4 = 1013364453467$ es primo o no. se puede ver que dicho número pasa el test de Solovay-Strassen para las bases
    2, 3, 5, 7 y 11, por lo que tenemos altas probabilidades de que dicho número sea primo. Busquemos ahora un elemento primitivo de $m_4$.
    Para ello factorizamos primero el número $m_4 - 1 = 1013364453466 = 2 \cdot 506682226733$. 
    
    Veamos si $m_{4,1} = 506682226733$ es primo o no.
    Aplicamos de nuevo el test de composición de Fermat y obtenemos que $2^{m_{4,1} -1} \not\equiv 1 \pmod{m_{4,1}}$ (algoritmo de exponenciación rápida) 
    con lo que obtenemos un certificado de composición. Busquemos ahora sus factores primos mediante el algoritmo $\rho$ de Polard, con el que 
    obtenemos un divisor 73 en solo 6 iteraciones, quedando por tanto $506682226733 = 73 \cdot 6940852421$. 
    
    Tomando ahora $m_{4,2} = 6940852421$
    veamos si es primo. Vemos que pasa el test de Solovay-Strassen para las bases 2, 3, 5, 7 y 11, por lo que tenemos altas probabilidades de que dicho número sea primo.
    Busquemos ahora un elemento primitivo de $m_{4,2}$. Para ello factorizamos primero el número $m_{4,2} - 1 = 6940852420 = 2^2 \cdot 5 \cdot 347042621$.
    
    Veamos si $m_{4,2,1} = 347042621$ es primo o no. Vemos que pasa el test de Solovay-Strassen para las bases 2, 3, 5, 7 y 11, por lo que tenemos altas 
    probabilidades de que dicho número sea primo. Aplicamos de nuevo el algoritmo de Lucas-Lehmer para ver si $m_{4,2,1}$ es primo o no. Para ello
    factorizamos $m_{4,2,1} - 1 =  2^2 \cdot 5 \cdot 17352131$, y veamos si $m_{4,2,1,1} = 17352131$ es primo o no. 
    
    Vemos que pasa el test de Solovay-Strassen
    para las bases 2, 3, 5, 7 y 11, por lo que tenemos altas probabilidades de que dicho número sea primo. Aplicamos de nuevo el algoritmo de Lucas-Lehmer
    para ver si $m_{4,2,1,1}$ es primo o no. Para ello factorizamos $m_{4,2,1,1} - 1 =  2 \cdot 5 \cdot 1735213$, y veamos si $m_{4,2,1,1,1} = 1735213$
    es primo o no. 

    Aplicando el test de composición de Fermat, obtenemos que $2^{m_{4,2,1,1,1} -1} \not\equiv 1 \pmod{m_{4,2,1,1,1}}$ (algoritmo de exponenciación rápida)
    con lo que obtenemos un certificado de composición. Aplicamos el algoritmo $\rho$ de Polard para obtener un divisor 19 en solo 4 iteraciones, quedando por tanto
    $1735213 = 19 \cdot 91327$. 
    
    Veamos de nuevo si $m_{4,2,1,1,2} = 91327$ es primo o no. Aplicando el test de composición de Fermat, obtenemos que
    $2^{m_{4,2,1,1,2} -1} \not\equiv 1 \pmod{m_{4,2,1,1,2}}$ (algoritmo de exponenciación rápida) con lo que obtenemos un certificado de composición.
    Aplicamos el algoritmo $\rho$ de Polard para obtener un divisor 271 en solo 9 iteraciones, quedando por tanto $91327 = 271 \cdot 337$, quedando así completamente 
    descompuesto.\\ 
    
    Tenemos por tanto $m_{4,2,1,1} - 1 = 2 \cdot 5 \cdot 5 \cdot 19 \cdot 271 \cdot 337$, por lo que estamos preparados para buscar un elemento primitivo de $m_{4,2,1,1} = 17352131$.
    \begin{enumerate}
        \item[$\bullet$] $2^{m_{4,2,1,1} -1} \equiv 1 \pmod{m_{4,2,1,1}}$
        \item[$\bullet$] $2^{(m_{4,2,1,1} -1)/2} \not\equiv 1 \pmod{m_{4,2,1,1}}$
        \item[$\bullet$] $2^{(m_{4,2,1,1} -1)/5} \not\equiv 1 \pmod{m_{4,2,1,1}}$
        \item[$\bullet$] $2^{(m_{4,2,1,1} -1)/19} \not\equiv 1 \pmod{m_{4,2,1,1}}$
        \item[$\bullet$] $2^{(m_{4,2,1,1} -1)/271} \not\equiv 1 \pmod{m_{4,2,1,1}}$
        \item[$\bullet$] $2^{(m_{4,2,1,1} -1)/337} \not\equiv 1 \pmod{m_{4,2,1,1}}$
    \end{enumerate}
    Y así hemos obtenido un elemento primitivo, luego podemos afirmar que $m_{4,2,1,1} = 17352131$ es primo.

    Con este resultado obtenemos la factorización en primos de $m_{4,2,1} - 1 = 2^2 \cdot 5 \cdot 17352131$, por lo que estamos preparados para buscar un elemento primitivo de $m_{4,2,1} = 347042621$.
    \begin{enumerate}
        \item[$\bullet$] $3^{m_{4,2,1} -1} \equiv 1 \pmod{m_{4,2,1}}$
        \item[$\bullet$] $3^{(m_{4,2,1} -1)/2} \not\equiv 1 \pmod{m_{4,2,1}}$
        \item[$\bullet$] $3^{(m_{4,2,1} -1)/5} \not\equiv 1 \pmod{m_{4,2,1}}$ 
        \item[$\bullet$] $3^{(m_{4,2,1} -1)/17352131} \not\equiv 1 \pmod{m_{4,2,1}}$
    \end{enumerate}
    Y así hemos obtenido un elemento primitivo, luego podemos afirmar que $m_{4,2,1} = 347042621$ es primo.

    Con este resultado obtenemos la factorización en primos de $m_{4,2} - 1 = 2^2 \cdot 5 \cdot 347042621$, por lo que estamos preparados para buscar un elemento primitivo de $m_{4,2} = 6940852421$.
    \begin{enumerate}
        \item[$\bullet$] $3^{m_{4,2} -1} \equiv 1 \pmod{m_{4,2}}$
        \item[$\bullet$] $3^{(m_{4,2} -1)/2} \not\equiv 1 \pmod{m_{4,2}}$
        \item[$\bullet$] $3^{(m_{4,2} -1)/5} \not\equiv 1 \pmod{m_{4,2}}$ 
        \item[$\bullet$] $3^{(m_{4,2} -1)/347042621} \not\equiv 1 \pmod{m_{4,2}}$
    \end{enumerate}
    Y así hemos obtenido un elemento primitivo, luego podemos afirmar que $m_{4,2} = 6940852421$ es primo.

    Con este resultado obtenemos la factorización en primos de $m_{4} - 1 = 2 \cdot 73 \cdot 6940852421$, por lo que estamos preparados para buscar un elemento primitivo de $m_{4} = 1013364453467$.
    \begin{enumerate}
        \item[$\bullet$] $2^{m_{4} -1} \equiv 1 \pmod{m_{4}}$
        \item[$\bullet$] $2^{(m_{4} -1)/2} \not\equiv 1 \pmod{m_{4}}$
        \item[$\bullet$] $2^{(m_{4} -1)/73} \not\equiv 1 \pmod{m_{4}}$
        \item[$\bullet$] $2^{(m_{4} -1)/6940852421} \not\equiv 1 \pmod{m_{4}}$
    \end{enumerate}
    Y así hemos obtenido un elemento primitivo, luego podemos afirmar que $m_{4} = 1013364453467$ es primo.
    Con ello hemos obtenido finalmente una factorización total del número $r = n+1$ en factores primos, que son los siguientes:
    \begin{enumerate}
        \item[$p_1$ =] $2^3$
        \item[$p_2$ =] 4349
        \item[$p_3$ =] 62347
        \item[$p_4$ =] 1924630699 
        \item[$p_5$ =] 1013364453467 
    \end{enumerate}
    Calculemos a continuación $U_{r/p}$ para cada $p$ en la lista anterior:
    \begin{enumerate}
        \item[-] $U_{r/2} = U_ {2115329543396028934802646178396} = 0$ % Cogido de la tabla
        \item[-] $U_{r/4349} = U_ {972788936949197026811978008} = $
        \item[-] $U_{r/62347} = U_ {67856658488653148822000936} = $
        \item[-] $U_{r/1924630699} = U_ {2198166686726042848808} = $
        \item[-] $U_{r/1013364453467} = U_ {4174864307029719976} = $
    \end{enumerate}

    Finalmente, como vimos anteriormente, tenemos que $U_{r/2} \equiv 0$, lo que nos asegura que el rango de Lucas $\omega(n) \neq n+1$, por lo
    que concluimos que para los valores de $Q$ y $P$ dados, no podemos asegurar la primalidad de nuestro número $n$.


\end{document}