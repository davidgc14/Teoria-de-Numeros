\documentclass[fleqn]{article}

%\pgfplotsset{compat=1.17}

\usepackage{mathexam}
\usepackage{amsmath}
\usepackage{amsfonts}
\usepackage{graphicx}
\usepackage{systeme}
\usepackage{microtype}
\usepackage{multirow}
\usepackage{pgfplots}
\usepackage{listings}
\usepackage{tikz}
\usepackage{dsfont} %Numeros reales, naturales...
\usepackage{cancel}
\usepackage{babel}

%\graphicspath{{images, }}
\newcommand*{\QED}{\hfill\ensuremath{\square}}

%Estructura de ecuaciones
\setlength{\textwidth}{15cm} \setlength{\oddsidemargin}{5mm}
\setlength{\textheight}{23cm} \setlength{\topmargin}{-1cm}



\author{David García Curbelo}
\title{Teoría de Números y Criptografía}

\pagestyle{empty}


\def\R{\mathds{R}}
\def\Z{\mathds{Z}}
\def\N{\mathds{N}}
\def\Q{\mathds{Q}}

\def\sup{$^2$}

\def\next{\quad \Rightarrow \quad}


\begin{document}
    \begin{center}
        \LARGE{\textbf{Ejercicio 7}} \\
        \Large{David García Curbelo} \\
    \end{center}

    \vspace{1cm}
    Toma tu número $n=4230659086792057869605292356791$ de la lista publicada para el ejercicio 3. Sea $d$ el 
    primer elemento de la sucesión $5, -7, 9, -11, 13, \dots$ que satisface que el símbolo de Jacobi es $(d|n) = -1$.


    \textbf{Apartado I. \textit{Con $P=1$, $Q = (1-d)/4$, define el e.c. $\alpha$ y sus sucesiones de Lucas asociadas.}}\\
    Calculamos primero el valor de $d$ mediante el símbolo de Jacobi:

    \begin{enumerate}
        \item[-] $\left(\frac{5}{n}\right) = 1$
        \item[-] $\left(\frac{-7}{n}\right) = 1$
        \item[-] $\left(\frac{9}{n}\right) = 1$
        \item[-] $\left(\frac{-11}{n}\right) = 1$
        \item[-] $\left(\frac{13}{n}\right) = -1$   
    \end{enumerate}

    Hemos encontrado el valor de $d = 13$ que nos interesa. Así podemos hayar la forma explícita de $P = 1$ y $Q = -3$.
    De la misma manera podemos hayar la forma explícita de $\Delta = P^2 - 4Q = 13$ y por tanto obtener
    $\alpha = \frac{P + \sqrt{\Delta}}{2} = \frac{1 + \sqrt{13}}{2}$. \\
    Las sucesiones de lucas asociadas son las siguientes:
    \begin{enumerate}
        \item[$\bullet$] $V_n = P \cdot V_{n-1} - Q \cdot V_{n-2} = V_{n-1} + 3 \cdot V_{n-2}$ \\ 
        \item[$\bullet$] $U_n = P \cdot U_{n-1} - Q \cdot U_{n-2} = U_{n-1} + 3 \cdot U_{n-2}$ \\
    \end{enumerate}
    Con $V_0 = 2$, $V_1 = P$, $U_0 = 0$, $U_1 = 1$.

    \newpage
    \textbf{Apartado II. \textit{Si $n$ primo, ¿Qué debería pasarle a $V_r$, $U_r$, módulo $n$? ¿Y a $V_{r/2}$, $U_{r/2}$?
            Calcula los términos $V_r$, $U_r$, $V_{r/2}$, $U_{r/2}$ módulo $n$, de las sucesiones de Lucas.
            ¿Tu $n$ verifica el TPF para el entero cuadrático $\alpha$?}}\\
    Si tomamos $n$ suponiendo que es primo, por la tercera versión del TPF para elementos cuadráticos tenemos que, como 
    $\left(\frac{\Delta}{n}\right) = -1$ por definición, tienen que cumplirse las siguientes ecuaciones:\\ 
    $\left\{
    \begin{aligned}
        U_{n-\left(\frac{\Delta}{n}\right)} \equiv 0 \pmod{n} &\next U_{n+1} \equiv 0 \pmod{n} \\
        V_{n-\left(\frac{\Delta}{n}\right)} \equiv 2Q \pmod{n} &\next V_{n+1} \equiv -6 \pmod{n} \\
    \end{aligned}
    \right.$


    
    \newpage
    \textbf{Apartado III. \textit{Factoriza $r = n+1$ y para cada factor primo $p$ suyo, calcula $U_{r/p}$.
            ¿Cuál es el rango de Lucas $w(n)$? ¿Qué deduces sobre la primalidad de tu $n$?}} \\
    Factorizamos $r = n+1 = 4230659086792057869605292356792 = 2^3 \cdot 528832385849007233700661544599$. Desconocemos si
    $m_1 = 528832385849007233700661544599$ es primo o no. Para saberlo aplicamos el test de composición de Fermat y obtenemos que 
    $2^{m_1 -1} \not\equiv 1 \pmod{m_1}$ (algoritmo de exponenciación rápida) con lo que obtenemos un certificado de composición. Ahora factorizando mediante el algoritmo $\rho$
    de Polard, obtenemos un divisor 4349 en 46 iteraciones, quedando por tanto $528832385849007233700661544599 = 4349 \cdot 121598617118649628351497251$.

    Veamos a continuación si $m_2 = 121598617118649628351497251$ es primo o no. Para saberlo aplicamos el test de composición de Fermat y
    obtenemos que $2^{m_2 -1} \not\equiv 1 \pmod{m_2}$ (algoritmo de exponenciación rápida) con lo que obtenemos un certificado de composición. 
    Ahora factorizando mediante el algoritmo $\rho$ de Polard, obtenemos un divisor 62347 en 46 iteraciones, quedando por tanto $121598617118649628351497251 = 62347 \cdot 1950352336417945183433$.
    
    Repetimos el mismo proceso para $m_3 = 1950352336417945183433$. Aplicando el test de composición de Fermat, obtenemos que 
    $2^{m_3 -1} \not\equiv 1 \pmod{m_3}$ (algoritmo de exponenciación rápida) con lo que obtenemos un certificado de composición.
    Ahora factorizando mediante el algoritmo $\rho$ de Polard, obtenemos un divisor 1924630699 en 26395 iteraciones, quedando por tanto 
    $1950352336417945183433 = 1924630699 \cdot 1013364453467$.
    
    Veamos por último si $m_4 = 1013364453467$ es primo o no. se puede ver que dicho número pasa el test de Solovay-Strassen para las bases
    2, 3, 5, 7 y 11, por lo que tenemos altas probabilidades de que dicho número sea primo. Busquemos ahora un elemento primitivo de $m_4$.
    Para ello factorizamos primero el número $m_4 - 1 = 1013364453466 = 2 \cdot 506682226733$. 
    
    Veamos si $m_{4,1} = 506682226733$ es primo o no.
    Aplicamos de nuevo el test de composición de Fermat y obtenemos que $2^{m_{4,1} -1} \not\equiv 1 \pmod{m_{4,1}}$ (algoritmo de exponenciación rápida) 
    con lo que obtenemos un certificado de composición. Busquemos ahora sus factores primos mediante el algoritmo $\rho$ de Polard, con el que 
    obtenemos un divisor 73 en solo 6 iteraciones, quedando por tanto $506682226733 = 73 \cdot 6940852421$. 
    
    Tomando ahora $m_{4,2} = 6940852421$
    veamos si es primo. Vemos que pasa el test de Solovay-Strassen para las bases 2, 3, 5, 7 y 11, por lo que tenemos altas probabilidades de que dicho número sea primo.
    Busquemos ahora un elemento primitivo de $m_{4,2}$. Para ello factorizamos primero el número $m_{4,2} - 1 = 6940852420 = 2^2 \cdot 5 \cdot 347042621$.
    
    Veamos si $m_{4,2,1} = 347042621$ es primo o no. Vemos que pasa el test de Solovay-Strassen para las bases 2, 3, 5, 7 y 11, por lo que tenemos altas 
    probabilidades de que dicho número sea primo. Aplicamos de nuevo el algoritmo de Lucas-Lehmer para ver si $m_{4,2,1}$ es primo o no. Para ello
    factorizamos $m_{4,2,1} - 1 =  2^2 \cdot 5 \cdot 17352131$, y veamos si $m_{4,2,1,1} = 17352131$ es primo o no. 
    
    Vemos que pasa el test de Solovay-Strassen
    para las bases 2, 3, 5, 7 y 11, por lo que tenemos altas probabilidades de que dicho número sea primo. Aplicamos de nuevo el algoritmo de Lucas-Lehmer
    para ver si $m_{4,2,1,1}$ es primo o no. Para ello factorizamos $m_{4,2,1,1} - 1 =  2 \cdot 5 \cdot 1735213$, y veamos si $m_{4,2,1,1,1} = 1735213$
    es primo o no. 

    Aplicando el test de composición de Fermat, obtenemos que $2^{m_{4,2,1,1,1} -1} \not\equiv 1 \pmod{m_{4,2,1,1,1}}$ (algoritmo de exponenciación rápida)
    con lo que obtenemos un certificado de composición. Aplicamos el algoritmo $\rho$ de Polard para obtener un divisor 19 en solo 4 iteraciones, quedando por tanto
    $1735213 = 19 \cdot 91327$. 
    
    Veamos de nuevo si $m_{4,2,1,1,2} = 91327$ es primo o no. Aplicando el test de composición de Fermat, obtenemos que
    $2^{m_{4,2,1,1,2} -1} \not\equiv 1 \pmod{m_{4,2,1,1,2}}$ (algoritmo de exponenciación rápida) con lo que obtenemos un certificado de composición.
    Aplicamos el algoritmo $\rho$ de Polard para obtener un divisor 271 en solo 9 iteraciones, quedando por tanto $91327 = 271 \cdot 337$, quedando así completamente 
    descompuesto.\\ 
    
    Tenemos por tanto $m_{4,2,1,1} - 1 = 2 \cdot 5 \cdot 5 \cdot 19 \cdot 271 \cdot 337$, por lo que estamos preparados para buscar un elemento primitivo de $m_{4,2,1,1} = 17352131$.
    \begin{enumerate}
        \item[$\bullet$] $2^{m_{4,2,1,1} -1} \equiv 1 \pmod{m_{4,2,1,1}}$
        \item[$\bullet$] $2^{(m_{4,2,1,1} -1)/2} \not\equiv 1 \pmod{m_{4,2,1,1}}$
        \item[$\bullet$] $2^{(m_{4,2,1,1} -1)/5} \not\equiv 1 \pmod{m_{4,2,1,1}}$
        \item[$\bullet$] $2^{(m_{4,2,1,1} -1)/19} \not\equiv 1 \pmod{m_{4,2,1,1}}$
        \item[$\bullet$] $2^{(m_{4,2,1,1} -1)/271} \not\equiv 1 \pmod{m_{4,2,1,1}}$
        \item[$\bullet$] $2^{(m_{4,2,1,1} -1)/337} \not\equiv 1 \pmod{m_{4,2,1,1}}$
    \end{enumerate}
    Y así hemos obtenido un elemento primitivo, luego podemos afirmar que $m_{4,2,1,1} = 17352131$ es primo.

    Con este resultado obtenemos la factorización en primos de $m_{4,2,1} - 1 = 2^2 \cdot 5 \cdot 17352131$, por lo que estamos preparados para buscar un elemento primitivo de $m_{4,2,1} = 347042621$.
    \begin{enumerate}
        \item[$\bullet$] $3^{m_{4,2,1} -1} \equiv 1 \pmod{m_{4,2,1}}$
        \item[$\bullet$] $3^{(m_{4,2,1} -1)/2} \not\equiv 1 \pmod{m_{4,2,1}}$
        \item[$\bullet$] $3^{(m_{4,2,1} -1)/5} \not\equiv 1 \pmod{m_{4,2,1}}$ 
        \item[$\bullet$] $3^{(m_{4,2,1} -1)/17352131} \not\equiv 1 \pmod{m_{4,2,1}}$
    \end{enumerate}
    Y así hemos obtenido un elemento primitivo, luego podemos afirmar que $m_{4,2,1} = 347042621$ es primo.

    Con este resultado obtenemos la factorización en primos de $m_{4,2} - 1 = 2^2 \cdot 5 \cdot 347042621$, por lo que estamos preparados para buscar un elemento primitivo de $m_{4,2} = 6940852421$.
    \begin{enumerate}
        \item[$\bullet$] $3^{m_{4,2} -1} \equiv 1 \pmod{m_{4,2}}$
        \item[$\bullet$] $3^{(m_{4,2} -1)/2} \not\equiv 1 \pmod{m_{4,2}}$
        \item[$\bullet$] $3^{(m_{4,2} -1)/5} \not\equiv 1 \pmod{m_{4,2}}$ 
        \item[$\bullet$] $3^{(m_{4,2} -1)/347042621} \not\equiv 1 \pmod{m_{4,2}}$
    \end{enumerate}
    Y así hemos obtenido un elemento primitivo, luego podemos afirmar que $m_{4,2} = 6940852421$ es primo.

    Con este resultado obtenemos la factorización en primos de $m_{4} - 1 = 2 \cdot 73 \cdot 6940852421$, por lo que estamos preparados para buscar un elemento primitivo de $m_{4} = 1013364453467$.
    \begin{enumerate}
        \item[$\bullet$] $2^{m_{4} -1} \equiv 1 \pmod{m_{4}}$
        \item[$\bullet$] $2^{(m_{4} -1)/2} \not\equiv 1 \pmod{m_{4}}$
        \item[$\bullet$] $2^{(m_{4} -1)/73} \not\equiv 1 \pmod{m_{4}}$
        \item[$\bullet$] $2^{(m_{4} -1)/6940852421} \not\equiv 1 \pmod{m_{4}}$
    \end{enumerate}
    Y así hemos obtenido un elemento primitivo, luego podemos afirmar que $m_{4} = 1013364453467$ es primo.
    Con ello hemos obtenido finalmente una factorización total del número $r = n+1$ en factores primos, que son los siguientes:
    \begin{enumerate}
        \item[$p_1$ =] $2^3$
        \item[$p_2$ =] 4349
        \item[$p_3$ =] 62347
        \item[$p_4$ =] 1924630699 
        \item[$p_5$ =] 1013364453467 
    \end{enumerate}
    Calculemos a continuación $U_{r/p}$ para cada $p$ en la lista anterior.

\end{document}