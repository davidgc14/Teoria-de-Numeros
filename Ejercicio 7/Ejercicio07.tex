\documentclass[fleqn]{article}

%\pgfplotsset{compat=1.17}

\usepackage{mathexam}
\usepackage{amsmath}
\usepackage{amsfonts}
\usepackage{graphicx}
\usepackage{systeme}
\usepackage{microtype}
\usepackage{multirow}
\usepackage{pgfplots}
\usepackage{listings}
\usepackage{tikz}
\usepackage{dsfont} %Numeros reales, naturales...
\usepackage{cancel}
\usepackage{babel}

%\graphicspath{{images, }}
\newcommand*{\QED}{\hfill\ensuremath{\square}}

%Estructura de ecuaciones
\setlength{\textwidth}{15cm} \setlength{\oddsidemargin}{5mm}
\setlength{\textheight}{23cm} \setlength{\topmargin}{-1cm}



\author{David García Curbelo}
\title{Teoría de Números y Criptografía}

\pagestyle{empty}


\def\R{\mathds{R}}
\def\Z{\mathds{Z}}
\def\N{\mathds{N}}
\def\Q{\mathds{Q}}

\def\sup{$^2$}

\def\next{\quad \Rightarrow \quad}


\begin{document}
    \begin{center}
        \LARGE{\textbf{Ejercicio 7}} \\
        \Large{David García Curbelo} \\
    \end{center}

    \vspace{1cm}
    Toma tu número $n=4230659086792057869605292356791$ de la lista publicada para el ejercicio 3. Sea $d$ el 
    primer elemento de la sucesión $5, -7, 9, -11, 13, \dots$ que satisface que el símbolo de Jacobi es $(d|n) = -1$.


    \textbf{Apartado I. \textit{Con $P=1$, $Q = (1-d)/4$, define el e.c. $\alpha$ y sus sucesiones de Lucas asociadas.}}
    Calculamos primero el valor de $d$ mediante el símbolo de Jacobi:

    \begin{enumerate}
        \item[$d = 5$] $\left(\frac{5}{n}\right) = \left(\frac{n}{5}\right)$ por ser $5 \equiv 1 \pmod{4}$. Pero ahora
                        vemos que $n \equiv 1 \pmod{5}$, luego tenemos $\left(\frac{n}{5}\right) = \left(\frac{1}{5}\right) = 1$. 
        \item[$d = -7$] $\left(\frac{-7}{n}\right) = \left(\frac{-1}{n}\right) \left(\frac{7}{n}\right) = -(-1)^{(n-1)/2}\left(\frac{7}{n}\right)
                        = \left(\frac{7}{n}\right)$ por ser $7 \equiv 3 \pmod{4}$. Pero ahora vemos que $n \equiv 1 \pmod{7}$, luego tenemos 
                        $\left(\frac{n}{-7}\right) = \left(\frac{1}{7}\right) = 1$.
        \item[$d = 9$] $\left(\frac{9}{n}\right) = \left(\frac{n}{9}\right)$ por ser $9 \equiv 1 \pmod{4}$. Pero ahora
                        vemos que $n \equiv 7 \pmod{9}$, luego tenemos $\left(\frac{n}{9}\right) = \left(\frac{7}{9}\right)$. Ahora,
                        $\left(\frac{7}{9}\right) = -\left(\frac{9}{7}\right)$ porque $7 \equiv 3 \pmod{4}$. Vemos que
                        $9 \equiv 2 \pmod{7}$, luego tenemos $-\left(\frac{9}{7}\right) = -\left(\frac{2}{7}\right)$. Comprobamos a 
                        continuación que $7 \equiv -1 \pmod{8}$, luego concluimos $-\left(\frac{2}{7}\right) = -1$.
    \end{enumerate}

    Hemos encontrado el valor de $d = 9$ que nos interesa. Así podemos hayar la forma explícita de $P = 1$ y $Q = -2$.
    De la misma manera podemos hayar la forma explícita de $\Delta = P^2 - 4Q = 9$ y $\alpha = P + Q$.

    \newpage
    \textbf{Apartado II. \textit{Si $n$ primo, ¿Qué debería pasarle a $V_r$, $U_r$, módulo $n$? ¿Y a $V_{r/2}$, $U_{r/2}$?
            Calcula los términos $V_r$, $U_r$, $V_{r/2}$, $U_{r/2}$ módulo $n$, de las sucesiones de Lucas.
            ¿Tu $n$ verifica el TPF para el entero cuadrático $\alpha$?}}

    
    \newpage
    \textbf{Apartado III. \textit{Factoriza $r = n+1$ y para cada factor primo $p$ suyo, calcula $U_{r/p}$.
            ¿Cuál es el rango de Lucas $w(n)$? ¿Qué deduces sobre la primalidad de tu $n$?}}
            



\end{document}