\documentclass[fleqn]{article}

%\pgfplotsset{compat=1.17}

\usepackage{mathexam}
\usepackage{amsmath}
\usepackage{amsfonts}
\usepackage{graphicx}
\usepackage{systeme}
\usepackage{microtype}
\usepackage{multirow}
\usepackage{pgfplots}
\usepackage{listings}
\usepackage{tikz}
\usepackage{dsfont} %Numeros reales, naturales...
\usepackage{cancel}
\usepackage{babel}

%\graphicspath{{images/}}
\newcommand*{\QED}{\hfill\ensuremath{\square}}

%Estructura de ecuaciones
\setlength{\textwidth}{15cm} \setlength{\oddsidemargin}{5mm}
\setlength{\textheight}{23cm} \setlength{\topmargin}{-1cm}



\author{David García Curbelo}
\title{Teoría de Números y Criptografía}

\pagestyle{empty}


\def\R{\mathds{R}}
\def\Z{\mathds{Z}}
\def\N{\mathds{N}}
\def\Q{\mathds{Q}}

\def\sup{$^2$}

\def\next{\quad \Rightarrow \quad}


\begin{document}
    \begin{center}
        \LARGE{\textbf{Ejercicio 6}} \\
        \Large{David García Curbelo} \\
    \end{center}

    \vspace{1cm}
    
    Sea $p = 4547$ el factor primo que tiene mayor período (en este caso, 58): \\ 


    \textbf{Apartado I. \textit{Calcula los convergentes de $\sqrt{p}$}} 

    \newpage
    \textbf{Apartado II. \textit{Calcula las soluciones de las ecuaciones de Pell, $x^2 - p \cdot y^2 = \pm 1$}}


    \newpage
    \textbf{Apartado III. \textit{Calcula las unidades del anillo de enteros cuadráticos $\Z[\sqrt{p}]$}}


    \newpage
    \textbf{Apartado IV. \textit{¿Es $\Z[\sqrt{p}]$ el anillo de enteros del cuerpo $\Q[\sqrt{p}]$?}}

\end{document}