\documentclass[fleqn]{article}

%\pgfplotsset{compat=1.17}

\usepackage{mathexam}
\usepackage{amsmath}
\usepackage{amsfonts}
\usepackage{graphicx}
\usepackage{systeme}
\usepackage{microtype}
\usepackage{multirow}
\usepackage{pgfplots}
\usepackage{listings}
\usepackage{tikz}
\usepackage{dsfont} %Numeros reales, naturales...
\usepackage{cancel}
\usepackage{babel}

%\graphicspath{{images, }}
\newcommand*{\QED}{\hfill\ensuremath{\square}}

%Estructura de ecuaciones
\setlength{\textwidth}{15cm} \setlength{\oddsidemargin}{5mm}
\setlength{\textheight}{23cm} \setlength{\topmargin}{-1cm}



\author{David García Curbelo}
\title{Teoría de Números y Criptografía}

\pagestyle{empty}


\def\R{\mathds{R}}
\def\Z{\mathds{Z}}
\def\N{\mathds{N}}
\def\Q{\mathds{Q}}

\def\sup{$^2$}

\def\next{\quad \Rightarrow \quad}


\begin{document}
    \begin{center}
        \LARGE{\textbf{Ejercicio 10}} \\
        \Large{David García Curbelo} \\
    \end{center}

    \vspace{1cm}
    Toma tu número $p = 45352609$ de la lista publicada para este ejercicio.

    \textbf{Apartado I. \textit{Calcula el símbolo de Jacobi $\left(\frac{-11}{p}\right)$. Si sale 1, usa el algoritmo de 
                                Tonelli-Shanks para hallar soluciones a la congruencia $x^2 \equiv -11 \pmod{p}$.}}\\
    
    Calculamos el símbolo de Jacobi: 
    $$\left(\frac{-11}{p}\right) = \left(\frac{-1}{p}\right) \left(\frac{-11}{p}\right) = \left(\frac{p}{11}\right) 
    = \left(\frac{5}{11}\right) = 1$$
    Procedemos a calcular las soluciones de la ecuación $x^2 \equiv -1 \pmod{p}$ mediante el algoritmo de Tonelli-Shanks.
    Para ello primero necesitamos estudiar la primalidad del número $p = 45352609$, para el que trataremos de encontrar un
    elemento primitivo. Vemos primeramente que $p$ pasa el test de Fermat para las bases $a = 2,3,5,7,11$, cumpliendo
    $a^{p-1} \equiv a \pmod{p}$, por lo que tenemos altas probabilidades de primalidad. Procedemos a buscar mediante el algoritmo
    de Lucas-Lehmer un elemento primitivo de $p$. Para ello primero factorizamos $p - 1 = 2^5 \cdot 3 \cdot 7 \cdot 67489$.
    
    Veamos si $p_1 = 67489$ es primo. Dicho número pasa el test de Fermat para las bases $a = 2,3,5,7,11$, cumpliendo
    $a^{p_1-1} \equiv a \pmod{p_1}$, por lo que tenemos altas probabilidades de primalidad. Factorizamos por tanto $p_1 - 1 $
    para poder aplicar el algoritmo de Lucas-Lehmer para $p_1$, con lo que obtenemos que $p_1 - 1 = 2^5 \cdot 3 \cdot 703$.
    Vemos que 703 no se encuentra en la lista de primos, luego aplicando el algoritmo $\rho$ de Polard obtenemos una factorización 
    completa de  $p_1 - 1 = 2^5 \cdot 3 \cdot 19 \cdot 37$. Por tanto estamos en condiciones de buscar un elemento primitivo para 
    $p_1$: 

    \begin{itemize}
        \item $23^{(p_1 - 1)} \equiv 1 \pmod{p_1}$
        \item $23^{(p_1 - 1)/2} \not\equiv 1 \pmod{p_1}$    
        \item $23^{(p_1 - 1)/3} \not\equiv 1 \pmod{p_1}$
        \item $23^{(p_1 - 1)/19} \not\equiv 1 \pmod{p_1}$
        \item $23^{(p_1 - 1)/37} \not\equiv 1 \pmod{p_1}$
    \end{itemize}

    Con lo que hemos encontrado un elemento primitivo, y por tanto tenemos certificado de primalidad de $p_1 = 67489$.
    Tenemos factorizado por tanto nuestro número $p-1 = 2^5 \cdot 3 \cdot 5 \cdot 7 \cdot 67489$ en factores primos, y estamos en condiciones de buscar un elemento 
    primitivo para $p$:

    \begin{itemize}
        \item $19^{(p - 1)} \equiv 1 \pmod{p}$
        \item $19^{(p - 1)/2} \not\equiv 1 \pmod{p}$    
        \item $19^{(p - 1)/3} \not\equiv 1 \pmod{p}$
        \item $19^{(p - 1)/7} \not\equiv 1 \pmod{p}$
        \item $19^{(p - 1)/67489} \not\equiv 1 \pmod{p}$
    \end{itemize}

    Con lo que hemos encontrado un elemento primitivo, y por tanto tenemos certificado de primalidad de $p = 45352609$, como andábamos buscando.
    
    Además comprobamos que $p \equiv 1 \pmod{8}$, por tanto necesitamos de un algoritmo especial, en nuestro caso del algoritmo de Tonelli-Shanks.
    Para ello, factorizamos $p = 2^5 \cdot 1417269$, luego tenemos que el algoritmo tendrá a lo sumo 5 pasos.

    Tomamos un número que no sea residuo cuadrático, por lo que para $n=19$ tenemos que $\left(\frac{19}{p}\right) = -1$, luego 19 no es residuo cuadrático.
    Entonces, un generador del 2-subgrupo de Sylow $ G \cong \Z_{2^{5}}$, viene dado por:
    \begin{itemize}
        \item   $z = n^q \equiv 19^{1417269} \equiv 26515681 \pmod{p}$\\
                $r = (-11)^{(q+1)/2} \equiv 11370107 \pmod{p}$\\
                Notemos ahora que $t = (-11)^q \equiv 196563 \not\equiv -1 \pmod{p}$, luego tenemos que $r$ no es raíz de la ecuación buscada.
                Calculamos ahora $O(t)$, que es un divisor de $2^{e-1}=16$. Luego tenemos que $t^{2^2} \equiv -1 \pmod{p} $, y por tanto
                obtenemos que $O(t) = 2^3$.
        \item   Lamamos ahora $b = z^{2^{e-i-1}} \equiv 23477766 \pmod{p} \not\equiv 1$ \\
                Definimos $t_1 = t \cdot b^2 \equiv 41942710 \pmod{p}$\\
                $r_1 = r \cdot b \equiv 25961315 \pmod{p}$\\
        \item   Definimos $t_2 = t_1 \cdot b^2 \equiv 6231274 \pmod{p} \not\equiv 1$\\
                $r_2 = r_1 \cdot b \equiv 11123330 \pmod{p}$\\
        \item   Definimos $t_3 = t_2 \cdot b^2 \equiv -1 \pmod{p} \not\equiv 1$\\
                $r_3 = r_2 \cdot b \equiv 3748274 \pmod{p}$\\
        \item   Definimos $t_4 = t_3 \cdot b^2 \equiv 45156046 \pmod{p} \not\equiv 1$\\
                $r_4 = r_3 \cdot b \equiv 31187509 \pmod{p}$\\
        \item   Definimos $t_5 = t_4 \cdot b^2 \equiv 3409899 \pmod{p} \not\equiv 1$\\
                $r_5 = r_4 \cdot b \equiv 18849057 \pmod{p}$\\
        \item   Definimos $t_6 = t_5 \cdot b^2 \equiv 39121335 \pmod{p} \not\equiv 1$\\
                $r_6 = r_5 \cdot b \equiv 43525646 \pmod{p}$\\
        \item   Definimos $t_7 = t_6 \cdot b^2 \equiv 1 \pmod{p}$\\
                $r_7 = r_6 \cdot b \equiv 36504054 \pmod{p}$\\
    \end{itemize}

    Hemos encontrado por fin las soluciones de la congruencia $x^2 \equiv -11 \pmod{p}$, las cuales vienen dadas por:
    \begin{itemize}
        \item $x_1 = r_7 = 36504054 $
        \item $x_2 = p - r_7 = 8848555$
    \end{itemize}


    \newpage
    \textbf{Apartado II. \textit{Usa una de estas soluciones para factorizar el ideal principal, 
                                $(p) = (p, n+ \sqrt{-11})(p, n+ \sqrt{-11})$ como producto de dos ideales.}}\\
    Tomando la solución impar $x_2 = 8848555$, y teniendo en cuenta que -11 no divide a nuestro primo p, tenemos que los ideales
    $\left(p,8848555 + \sqrt{-11}\right)$ y $\left(p,8848555 - \sqrt{-11}\right)$ son primos entre sí:
    $$ (p) = \left(p,8848555 + \sqrt{-11}\right)\left(p,8848555 - \sqrt{-11}\right)$$

    \newpage
    \textbf{Apartado III. \textit{Aplica el algoritmo de Conachia-Smith modificando a $2p$ y $n$ para encontrar
                                una solución a la ecuación diofántica $4p = x^2 + 11y^2 $ y la usas para encontrar 
                                una factorización de $p$ en a.e. del cuerpo $\Q[\sqrt{p}]$.}} \\
    Para ello usaremos la solución impar del apartado 1, $x_2 = 8848555$, y aplicaremos el algoritmo de Conachia-Smith.

    \begin{itemize}
        \item $90705218=8848555 \cdot 10+2219668$
        \item $8848555=2219668 \cdot 3+2189551$
        \item $2219668=2189551 \cdot 1+30117$
        \item $2189551=30117 \cdot 72+21127$
        \item $30117=21127 \cdot 1+8990$
    \end{itemize}
    Por tanto, después de 10 divisores tenemos $x = 8990$ y $x^2 + 11 \cdot y^2 = 181410436$, con lo que despejando obtenemos
    la solución $y = 3024$. Encontremos ahora una factorización de $p$ en a.e. del cuerpo $\Q[\sqrt{p}]$:
    $$p = \frac{x^2 + 11 \cdot y^2}{4} = \frac{8990^2 + 11 \cdot 3024^2}{4} \next  p = \left(\frac{8990 + 3024 \sqrt{-11}}{2}\right) \left(\frac{8990 - 3024 \sqrt{-11}}{2}\right)$$

    \newpage
    \textbf{Apartado IV. \textit{¿Son principales sus ideales $(p, n+ \sqrt{-11})$ y $(p, n+ \sqrt{-11})$ ?}}\\
    Por resultados vistos en teoría (ver Teorema 17), $\Q(\sqrt{-11})$ es un dominio de ideales principales, además
    de que $\left(p,8848555 + \sqrt{-11}\right)$ y $\left(p,8848555 - \sqrt{-11}\right)$ son ideales suyos. Por tanto 
    podemos concluir que estos son ideales principales.



\end{document}