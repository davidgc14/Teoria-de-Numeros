\documentclass[fleqn]{article}

%\pgfplotsset{compat=1.17}

\usepackage{mathexam}
\usepackage{amsmath}
\usepackage{amsfonts}
\usepackage{graphicx}
\usepackage{systeme}
\usepackage{microtype}
\usepackage{multirow}
\usepackage{pgfplots}
\usepackage{listings}
\usepackage{tikz}
\usepackage{dsfont} %Numeros reales, naturales...
\usepackage{cancel}
\usepackage{babel}

%\graphicspath{{images, }}
\newcommand*{\QED}{\hfill\ensuremath{\square}}

%Estructura de ecuaciones
\setlength{\textwidth}{15cm} \setlength{\oddsidemargin}{5mm}
\setlength{\textheight}{23cm} \setlength{\topmargin}{-1cm}



\author{David García Curbelo}
\title{Teoría de Números y Criptografía}

\pagestyle{empty}


\def\R{\mathds{R}}
\def\Z{\mathds{Z}}
\def\N{\mathds{N}}
\def\Q{\mathds{Q}}

\def\sup{$^2$}

\def\next{\quad \Rightarrow \quad}


\begin{document}
    \begin{center}
        \LARGE{\textbf{Ejercicio 10}} \\
        \Large{David García Curbelo} \\
    \end{center}

    \vspace{1cm}
    Toma tu número $p = 45352609$ de la lista publicada para este ejercicio.

    \textbf{Apartado I. \textit{Calcula el símbolo de Jacobi $\left(\frac{-11}{p}\right)$. Si sale 1, usa el algoritmo de 
                                Tonelli-Shanks para hallar soluciones a la congruencia $x^2 \equiv -1 \pmod{p}$.}}\\
    

    \newpage
    \textbf{Apartado II. \textit{Usa una de estas soluciones para factorizar el ideal principal, 
                                $(p) = (p, n+ \sqrt{-11})(p, n+ \sqrt{-11}) como producto de dos ideales.$}}\\

    \newpage
    \textbf{Apartado III. \textit{Aplica el algoritmo de Conachia-Smith modificando a $2p$ y $n$ para encontrar
                                una solución a la ecuación diofántica $4p = x^2 + 11y^2 $ y la usas para encontrar 
                                una factorización de $p$ en a.e. del cuerpo $\Q[\sqrt{p}]$.}} \\


    \newpage
    \textbf{Apartado IV. \textit{¿Son principales sus ideales $(p, n+ \sqrt{-11})$ y $(p, n+ \sqrt{-11})$ ?}}\\

\end{document}