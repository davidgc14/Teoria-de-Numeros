\documentclass[fleqn]{article}

%\pgfplotsset{compat=1.17}

\usepackage{mathexam}
\usepackage{amsmath}
\usepackage{amsfonts}
\usepackage{graphicx}
\usepackage{systeme}
\usepackage{microtype}
\usepackage{multirow}
\usepackage{pgfplots}
\usepackage{listings}
\usepackage{tikz}
\usepackage{dsfont} %Numeros reales, naturales...
\usepackage{cancel}
\usepackage{babel}

%\graphicspath{{images, }}
\newcommand*{\QED}{\hfill\ensuremath{\square}}

%Estructura de ecuaciones
\setlength{\textwidth}{15cm} \setlength{\oddsidemargin}{5mm}
\setlength{\textheight}{23cm} \setlength{\topmargin}{-1cm}



\author{David García Curbelo}
\title{Teoría de Números y Criptografía}

\pagestyle{empty}


\def\R{\mathds{R}}
\def\Z{\mathds{Z}}
\def\N{\mathds{N}}
\def\Q{\mathds{Q}}

\def\sup{$^2$}

\def\next{\quad \Rightarrow \quad}


\begin{document}
    \begin{center}
        \LARGE{\textbf{Ejercicio 8}} \\
        \Large{David García Curbelo} \\
    \end{center}

    \vspace{1cm}

    \textbf{Preámbulo} \\
    Toma tu número $n=191871308917122834687961459636870046909$ de la lista publicada para el ejercicio 2.
    Escribe n en base 2, usa esas cifras para definir un polinomio, $f(x)$, donde tu bit más significativo
    defina el grado del polinomio $n$, el siguiente bit va multiplicado por $x^{n-1}$ y sucesivamente hasta 
    que el bit menos significativo sea el término independiente. El polinomio que obtienes es universal en el
    sentido de que tiene coeficientes en cualquier anillo.

    \newpage
    Sea $f(x)$ el polinomio que obtienes con coeficientes en $\Z$.\\ 

    \textbf{Apartado I. \textit{Toma $g(x) = f(x) \pmod{2}$ y haya el menor cuerpo de característica 2 que contenga
                                a todas las raíces de $g$. ¿Qué deduces sobre la irreducibilidad de $g(x)$ en $\Z_2[x]$ ?}}\\
    

    \newpage
    \textbf{Apartado II. \textit{Extrae la parte libre de cuadrados de $g(x)$ y le calculas su matriz de Berlekamp por
                                columnas. Resuelve el s.l. $(B - Id)X = 0$.}}\\

    \newpage
    \textbf{Apartado III. \textit{Aplica Berlekamp si es necesario recursivamente para hallar la descomposición en irreducibles de $g(x)$ en $\Z_2[x]$.}} \\


    \newpage
    \textbf{Apartado IV. \textit{Haz lo mismo para hallar la descomposición en irreducibles de $f(x) \pmod{3}$}}\\
    \newpage
    \textbf{Apartado V. \textit{¿Qué deduces sobre la reducibilidad de $f(x)$ en $\Z[x]$?}}\\

\end{document}