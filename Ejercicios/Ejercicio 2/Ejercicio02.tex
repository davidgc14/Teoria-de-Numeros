\documentclass[fleqn]{article}

%\pgfplotsset{compat=1.17}

\usepackage{mathexam}
\usepackage{amsmath}
\usepackage{amsfonts}
\usepackage{graphicx}
\usepackage{systeme}
\usepackage{microtype}
\usepackage{multirow}
\usepackage{pgfplots}
\usepackage{listings}
\usepackage{tikz}
\usepackage{dsfont} %Numeros reales, naturales...
\usepackage{cancel}
\usepackage{babel}

%\graphicspath{{images/}}
\newcommand*{\QED}{\hfill\ensuremath{\square}}

%Estructura de ecuaciones
\setlength{\textwidth}{15cm} \setlength{\oddsidemargin}{5mm}
\setlength{\textheight}{23cm} \setlength{\topmargin}{-1cm}



\author{David García Curbelo}
\title{Teoría de Números y Criptografía}

\pagestyle{empty}


\def\R{\mathds{R}}
\def\Z{\mathds{Z}}
\def\N{\mathds{N}}

\def\sup{$^2$}

\def\next{\quad \Rightarrow \quad}


\begin{document}
    \begin{center}
        \LARGE{\textbf{Ejercicio 2}} \\
        \Large{David García Curbelo} \\
    \end{center}

    \vspace{1cm}
    
    Dado tu número $n$ = 45352609 de la lista publicada: \\ 


    \textbf{Apartado I. \textit{Usa el algoritmo manual para calcular el símbolo de Jacobi $\left(\frac{p}{n}\right)$ para $p$ 
    cada uno de los 5 primeros primos.}} 

    Calculamos el símbolo de Jacobi usando las propiedades conocidas:

    \begin{enumerate}
        \item[$\bullet$] $ \left(\frac{2}{n}\right) $. Como vemos que $45352609 = 5669076 \cdot 8 + 1$, entonces tenemos que
                    $ n \equiv 1 \text{ mod } 8 $, y por tanto $ \left(\frac{2}{45352609}\right) = (-1)^{(45352609^2 - 1)/8} = 1$.
        
        \item[$\bullet$] $ \left(\frac{3}{n}\right) $. Como vemos que $45352609 = 11338152 \cdot 4 + 1$, entonces tenemos que
                    $\left(\frac{3}{45352609}\right) = \left(\frac{45352609}{3}\right) $. Ahora, viendo que $45352609 = 15117536 \cdot 3 + 1$,
                    y así $ 45352609 \equiv 1 \text{ mod } 3$, obtenemos $\left(\frac{45352609}{3}\right) = \left(\frac{1}{3}\right) = 1$.
        
        \item[$\bullet$] $ \left(\frac{5}{n}\right) $. Por el mismo proceso anterior tenemos que $ \left(\frac{5}{45352609}\right) = \left(\frac{45352609}{5}\right)$
                    por cumplirse que $45352609 = 11338152 \cdot 4 + 1$. Ahora, viendo que $45352609 = 9070521 \cdot 5 + 4$, y así $ 45352609 \equiv 4 \text{ mod } 5$,
                    con lo que tenemos $\left(\frac{45352609}{5}\right) = \left(\frac{4}{5}\right)$. \\
                    Vemos que 4 es un residuo cuadrático módulo 5, ya que para $x = 2$ se tiene $x^2 \equiv 4 \pmod 5$, luego tenemos que $\left(\frac{4}{5}\right) = 1$.

        \item[$\bullet$] $ \left(\frac{7}{n}\right) $. Por el mismo proceso anterior tenemos que $ \left(\frac{7}{45352609}\right) = \left(\frac{45352609}{7}\right)$
                    por cumplirse que $45352609 = 11338152 \cdot 4 + 1$. Ahora, viendo que $45352609 = 6478944 \cdot 7 + 1$, y así $ 45352609 \equiv 1 \text{ mod } 7$,
                    con lo que tenemos $\left(\frac{45352609}{7}\right) = \left(\frac{1}{7}\right) = 1$. 

        \item[$\bullet$] $ \left(\frac{11}{n}\right) $. Por el mismo proceso anterior tenemos que $ \left(\frac{11}{45352609}\right) = \left(\frac{45352609}{11}\right)$
                    por cumplirse que $45352609 = 11338152 \cdot 4 + 1$. Ahora, viendo que $45352609 = 4122964 \cdot 11 + 5$, y así $ 45352609 \equiv 5 \text{ mod } 11$,
                    con lo que tenemos $\left(\frac{45352609}{11}\right) = \left(\frac{5}{11}\right) = 1$. \\ 
                    De nuevo, vemos que 5 es un residuo cuadrático módulo 11, ya que para $x = 4$ se tiene $x^2 = 16 \equiv 5 \pmod11$, luego tenemos que 
                    $\left(\frac{5}{11}\right) = 1$.
    \end{enumerate}


    \newpage
    \textbf{Apartado II. \textit{Si para alguna de esas bases tu número sale posible primo de Fermat, comprueba si además es posible primo de Euler.}}
    Aplicamos el algoritmo de exponenciación rápida de izquierda a derecha para el cálculo de $ a^{45352608} \pmod45352609$, con $a$ cada una de las bases:


    \begin{center}
        \begin{tabular}{c}
            Base 2 \\
        \end{tabular}
        \begin{tabular}{c | c | c}
            Iteración & Exponente & Acumulado \\ \hline
            0 & 1 & 1 \\
            1 & 1 & 2 \\
            2 & 2 & 4 \\
            3 & 5 & 32 \\
            4 & 10 & 1024 \\
            5 & 21 & 2097152 \\
            6 & 43 & 45211876 \\
            7 & 86 & 32039765 \\
            8 & 173 & 20054929 \\
            9 & 346 & 42973822 \\
            10 & 692 & 27919048 \\
            11 & 1384 & 36563318 \\
            12 & 2768 & 42979486 \\
            13 & 5536 & 7197945 \\
            14 & 11072 & 45227515 \\
            15 & 22144 & 1858731 \\
            16 & 44289 & 19763918 \\
            17 & 88579 & 5588446 \\
            18 & 177158 & 15085336 \\
            19 & 354317 & 22777171 \\
            20 & 708634 & 3728254 \\
            21 & 1417269 & 12390911 \\
            22 & 2834538 & 39121335 \\
            23 & 5669076 & 3409899 \\
            24 & 11338152 & 45352608 \\
            25 & 22676304 & 1 \\
            26 & 45352608 & 1 \\ \hline
            && $ 2^{45352608} \equiv  1 \pmod{n}$
        \end{tabular}
        \newpage
        \begin{tabular}{c}
            Base 3 \\
        \end{tabular}
        \begin{tabular}{c | c | c}
            Iteración & Exponente & Acumulado \\ \hline
            0 & 1 & 1 \\
            1 & 1 & 3 \\
            2 & 2 & 9 \\
            3 & 5 & 243 \\
            4 & 10 & 59049 \\
            5 & 21 & 29253133 \\
            6 & 43 & 12007364 \\
            7 & 86 & 29870534 \\
            8 & 173 & 767839 \\
            9 & 346 & 38165530 \\
            10 & 692 & 22647345 \\
            11 & 1384 & 10967526 \\
            12 & 2768 & 33282599 \\
            13 & 5536 & 43972016 \\
            14 & 11072 & 2933206 \\
            15 & 22144 & 35395482 \\
            16 & 44289 & 21244663 \\
            17 & 88579 & 32821326 \\
            18 & 177158 & 7784461 \\
            19 & 354317 & 15045949 \\
            20 & 708634 & 40214907 \\
            21 & 1417269 & 22577962 \\
            22 & 2834538 & 45156046 \\
            23 & 5669076 & 41942710 \\
            24 & 11338152 & 45352608 \\
            25 & 22676304 & 1 \\
            26 & 45352608 & 1 \\ \hline
            && $ 3^{45352608} \equiv  1 \pmod{n} $
        \end{tabular}
        \newpage
        \begin{tabular}{c}
            Base 5 \\
        \end{tabular}
        \begin{tabular}{c | c | c}
            Iteración & Exponente & Acumulado \\ \hline
            0 & 1 & 1 \\
            1 & 1 & 5 \\
            2 & 2 & 25 \\
            3 & 5 & 3125 \\
            4 & 10 & 9765625 \\
            5 & 21 & 8587561 \\
            6 & 43 & 22401770 \\
            7 & 86 & 27169989 \\
            8 & 173 & 16918953 \\
            9 & 346 & 42206134 \\
            10 & 692 & 11791361 \\
            11 & 1384 & 16046682 \\
            12 & 2768 & 34835928 \\
            13 & 5536 & 33387032 \\
            14 & 11072 & 21012559 \\
            15 & 22144 & 31965521 \\
            16 & 44289 & 35636762 \\
            17 & 88579 & 3154889 \\
            18 & 177158 & 14268136 \\
            19 & 354317 & 32915580 \\
            20 & 708634 & 36738832 \\
            21 & 1417269 & 23477766 \\
            22 & 2834538 & 196563 \\
            23 & 5669076 & 41942710 \\
            24 & 11338152 & 45352608 \\
            25 & 22676304 & 1 \\
            26 & 45352608 & 1 \\ \hline
            && $5^{45352608} \equiv  1 \pmod{n}$
        \end{tabular}
        \newpage
        \begin{tabular}{c}
            Base 7 \\
        \end{tabular}
        \begin{tabular}{c | c | c}
            Iteración & Exponente & Acumulado \\ \hline
            0 & 1 & 1 \\
            1 & 1 & 7 \\
            2 & 2 & 49 \\
            3 & 5 & 16807 \\
            4 & 10 & 10359595 \\
            5 & 21 & 43322858 \\
            6 & 43 & 44022215 \\
            7 & 86 & 17276402 \\
            8 & 173 & 3564957 \\
            9 & 346 & 28907433 \\
            10 & 692 & 38249152 \\
            11 & 1384 & 15340494 \\
            12 & 2768 & 13740019 \\
            13 & 5536 & 41156504 \\
            14 & 11072 & 8426346 \\
            15 & 22144 & 33255669 \\
            16 & 44289 & 32506299 \\
            17 & 88579 & 45071844 \\
            18 & 177158 & 6150783 \\
            19 & 354317 & 15956636 \\
            20 & 708634 & 14191769 \\
            21 & 1417269 & 22577962 \\
            22 & 2834538 & 45156046 \\
            23 & 5669076 & 41942710 \\
            24 & 11338152 & 45352608 \\
            25 & 22676304 & 1 \\
            26 & 45352608 & 1 \\ \hline
            && $ 7^{45352608} \equiv 32441529 \pmod{n} $ 
        \end{tabular}

        \begin{tabular}{c}
            Base 11 \\
        \end{tabular}
        \begin{tabular}{c | c | c}
            Iteración & Exponente & Acumulado \\ \hline
            0 & 1 & 1 \\
            1 & 1 & 11 \\
            2 & 2 & 121 \\
            3 & 5 & 161051 \\
            4 & 10 & 41084862 \\
            5 & 21 & 33913782 \\
            6 & 43 & 4241627 \\
            7 & 86 & 19616829 \\
            8 & 173 & 4695901 \\
            9 & 346 & 4595994 \\
            10 & 692 & 1795850 \\
            11 & 1384 & 7843901 \\
            12 & 2768 & 27597522 \\
            13 & 5536 & 31785545 \\
            14 & 11072 & 29259672 \\
            15 & 22144 & 39742362 \\
            16 & 44289 & 44905957 \\
            17 & 88579 & 1408461 \\
            18 & 177158 & 39270861 \\
            19 & 354317 & 4207156 \\
            20 & 708634 & 36073034 \\
            21 & 1417269 & 45156046 \\
            22 & 2834538 & 41942710 \\
            23 & 5669076 & 45352608 \\
            24 & 11338152 & 1 \\
            25 & 22676304 & 1 \\
            26 & 45352608 & 1 \\ \hline
            && $ 11^{45352608} \equiv  12846800 \pmod{n} $
        \end{tabular}\\
    \end{center}

    Vemos que, por el algoritmo de la exponenciación rápida, $n$ ha pasado el test de Fermat para las cinco bases propuestas. 
    Veamos si dichas bases también pasan el test de Euler. Para poder decir que estamos ante un posible primo de Euler 
    para la base $a$ tiene que verificarse que 
    $$\left(\frac{a}{n} \right) \equiv a^{(n-1)/2} \pmod{n}, \quad \text{con} \quad n=45352609$$
    
    \begin{enumerate}
        \item[$\bullet$] $ \left(\frac{2}{45352609}\right) = 1 \equiv 2^{22676304} \pmod{45352609} \Rightarrow a = 2$ cumple el test de Euler. 
        \item[$\bullet$] $ \left(\frac{3}{45352609}\right) = 1 \equiv 3^{22676304} \pmod{45352609} \Rightarrow a = 3$ cumple el test de Euler. 
        \item[$\bullet$] $ \left(\frac{5}{45352609}\right) = 1 \equiv 5^{22676304} \pmod{45352609} \Rightarrow a = 5$ cumple el test de Euler. 
        \item[$\bullet$] $ \left(\frac{7}{45352609}\right) = 1 \equiv 7^{22676304} \pmod{45352609} \Rightarrow a = 7$ cumple el test de Euler.
        \item[$\bullet$] $ \left(\frac{11}{45352609}\right) = 1 \equiv 11^{22676304} \pmod{45352609} \Rightarrow a = 11$ cumple el test de Euler.
    \end{enumerate}

    Donde el valor de $a^{(n-1)/2} \pmod{n}$ lo hemos calculado con la penúltima iteración del algoritmo de exponenciación rápida para cada base.
    Con lo que tenemos que 45352609 es un posible primo de Euler para las bases 2, 3, 5, 7 y 11.


    \newpage
    \textbf{Apartado III. \textit{¿Es tu número $n$ pseudoprimo de Fermat o de Euler para alguna de las bases?}}

    Como hemos visto en el apartado anterior, las bases 2, 3, 5, 7 y 11 pasan ambos test, luego podemos afirmar que $n$ es un presudoprimo
    tanto de Euler como de Fermat para cada una de estas tres bases. Este hecho nos indica que existe una gran probabilidad de que nuestro número
    $n$ sea efectivamente primo, y por tanto existe una baja probabilidad de que dicho número sea un pseudoprimo de Euler y de Fermat.


\end{document}