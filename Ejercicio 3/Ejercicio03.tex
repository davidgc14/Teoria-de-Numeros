\documentclass[fleqn]{article}

%\pgfplotsset{compat=1.17}

\usepackage{mathexam}
\usepackage{amsmath}
\usepackage{amsfonts}
\usepackage{graphicx}
\usepackage{systeme}
\usepackage{microtype}
\usepackage{multirow}
\usepackage{pgfplots}
\usepackage{listings}
\usepackage{tikz}
\usepackage{dsfont} %Numeros reales, naturales...
\usepackage{cancel}
\usepackage{babel}

%\graphicspath{{images/}}
\newcommand*{\QED}{\hfill\ensuremath{\square}}

%Estructura de ecuaciones
\setlength{\textwidth}{15cm} \setlength{\oddsidemargin}{5mm}
\setlength{\textheight}{23cm} \setlength{\topmargin}{-1cm}



\author{David García Curbelo}
\title{Teoría de Números y Criptografía}

\pagestyle{empty}


\def\R{\mathds{R}}
\def\Z{\mathds{Z}}
\def\N{\mathds{N}}

\def\sup{$^2$}

\def\next{\quad \Rightarrow \quad}


\begin{document}
    \begin{center}
        \LARGE{\textbf{Ejercicio 2}} \\
        \Large{David García Curbelo} \\
    \end{center}

    \vspace{1cm}
    
    Dado tu número $m = 4230659086792057869605292356791$  de la lista publicada: \\ 


    \textbf{Apartado I. \textit{Calcula $ a^{m-1} \pmod{m}$ para los 5 primeros primos.}} 

    \newpage
    \textbf{Apartado II. \textit{Calcula el test de Solovay-Strassen para los 5 primeros primos.}}

    \newpage
    \textbf{Apartado III. \textit{Calcula el test de Miller-Rabin para esas 5 bases.}}

    \newpage
    \textbf{Apartado IV. \textit{¿Qué deduces sobre la primalidad de tu número?}}


\end{document}