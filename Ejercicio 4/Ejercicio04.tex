\documentclass[fleqn]{article}

%\pgfplotsset{compat=1.17}

\usepackage{mathexam}
\usepackage{amsmath}
\usepackage{amsfonts}
\usepackage{graphicx}
\usepackage{systeme}
\usepackage{microtype}
\usepackage{multirow}
\usepackage{pgfplots}
\usepackage{listings}
\usepackage{tikz}
\usepackage{dsfont} %Numeros reales, naturales...
\usepackage{cancel}
\usepackage{babel}

%\graphicspath{{images/}}
\newcommand*{\QED}{\hfill\ensuremath{\square}}

%Estructura de ecuaciones
\setlength{\textwidth}{15cm} \setlength{\oddsidemargin}{5mm}
\setlength{\textheight}{23cm} \setlength{\topmargin}{-1cm}



\author{David García Curbelo}
\title{Teoría de Números y Criptografía}

\pagestyle{empty}


\def\R{\mathds{R}}
\def\Z{\mathds{Z}}
\def\N{\mathds{N}}

\def\sup{$^2$}

\def\next{\quad \Rightarrow \quad}


\begin{document}
    \begin{center}
        \LARGE{\textbf{Ejercicio 4}} \\
        \Large{David García Curbelo} \\
    \end{center}

    \vspace{1cm}
    
    Dado tu número $n = 45352609$ de la lista del ejercicio 2: \\ 


    \textbf{Apartado I. \textit{Factoriza $n-1$ aplicando el método $\rho$ de Polard. ¿Cuántas iteraciones necesitas?}} 

    Tenemos que $n-1 = 45352608 = 2^5 \cdot 3 \cdot 472423$. Ahora veamos si nuestro número 472423 es primo. Para ello
    aplicamos el test de composición de Fermat, con el que obtenemos que dicho número es compuesto, ya que
    $2^{472422} \equiv 64 \pmod{472423}$, luego tenemos certificado de composición. Tratamos ahora de factorizar 472423
    mediante el método de Polard usando como función aleatoria $f(x) = x^2 + 1$:

    \begin{center}
        \begin{tabular}{| c | c | c | c |}
            \hline Paso & $x$ & $y$ & mcd \\ \hline \hline
            0 & 1 & 1 & - \\ \hline
            1 & 2 & 5 & 1 \\ \hline 
            2 & 5 & 677 & 7 \\ \hline 
        \end{tabular}
    \end{center}

    Por lo que en dos iteraciones hemos encontrado que el número primo 7 es divisor de 472423, obteniendo así la factorización
    $n-1 = 45352608 = 2^5 \cdot 3 \cdot 7 \cdot 67489$. 
    Podemos ver que efectivamente el número 67489 pasa tanto el test de Miller-Rabin como el de Solovay-Strassen para
    los 5 primeros números primos, luego tenemos que la probabilidad de que efectivamente sea compuesto es de $4^{-5}$.



    \newpage
    \textbf{Apartado II. \textit{Si es necesario aplica recursivamente Lucas-Lehmer para certificar factores primos de n-1 mayores de 4 cifras.}}
   
    Tenemos un factor mayor de cuatro cifras: el número 67489. Veamos si es primo. Para ello necesitamos calcular la factorización en 
    primos de 67488, luego tenemos $67488 = 2^5 \cdot 3 \cdot 703$. Por la tabla de primos menores de 4 cifras, vemos que el número 703 no
    aparece en dicha tabla, luego es un número compuesto. Calculemos sus factores primos mediante el método $\rho$ de Polard:

    \begin{center}
        \begin{tabular}{| c | c | c | c |}
            \hline Paso & $x$ & $y$ & mcd \\ \hline \hline
            0 & 3 & 3 & - \\ \hline
            1 & 10 & 101 & 1 \\ \hline
            2 & 101 & 249 & 37 \\ \hline 
        \end{tabular}
    \end{center}

    Por lo que en 2 iteraciones hemos encontrado la factorización completa de $67488 = 2^5 \cdot 3 \cdot 19 \cdot 37$. Como ya conocemos
    todos sus factores primos (2, 3, 19 y 37), iniciamos la búsqueda de un elemento primitivo para tratar de probar que efectivamente el número
    67489 es primo. Para ello, aplicando el algoritmo de exponenciación rápida tenemos que, para $a=23$ se cumple:

    \begin{enumerate}
        \item $ 23^{67488} \equiv 1 \pmod{67489}$
        \item $ 23^{(67488)/2} \equiv -1 \pmod{67489}$
        \item $ 23^{(67488)/3} \equiv 13861 \pmod{67489}$
        \item $ 23^{(67488)/19} \equiv 52206 \pmod{67489}$
        \item $ 23^{(67488)/37} \equiv 17698 \pmod{67489}$
    \end{enumerate}

    Por lo que tenemos que 23 es un elemento primitivo de 67489 y por tanto tenemos un certificado de primalidad. Así hemos obtenido efectivamente
    la factorización en números primos de nuestro número original $n-1 = 45352608 = 2^5 \cdot 3 \cdot 7 \cdot 67489$.



    \newpage
    \textbf{Apartado III. \textit{Aplica Lucas-Lehmer para encontrar un certificado de primalidad de $n$.}}

    Para $n = 45352609$ tenemos $n-1 = 45352608 = 2^5 \cdot 3 \cdot 7 \cdot 67489$, luego conocemos todos sus factores primos (2, 3, 7 y 67489).
    Iniciamos la búsqueda de un elemento primitivo para tratar de probar que efectivamente nuestro número $n$ es primo. 
    Para ello, aplicando el algoritmo de exponenciación rápida tenemos que, para $a=19$ se cumple:

    \begin{enumerate}
        \item $ 19^{n-1} \equiv 1 \pmod{n}$
        \item $ 19^{(n-1)/2} \equiv -1 \pmod{n}$
        \item $ 19^{(n-1)/3} \equiv 3335204 \pmod{n}$
        \item $ 19^{(n-1)/19} \equiv 43001658 \pmod{n}$
        \item $ 19^{(n-1)/37} \equiv 14444632 \pmod{n}$
    \end{enumerate}

    Por lo que podemos confirmar que 19 es un elemento primitivo de $n$. Así hemos obtenido efectivamente un certificado de primalidad y por tanto concluimos
    que 45352609 es un número primo.
\end{document}